\chapter*{Kernthesen} 

    \vspace{8pt}
    
    \begin{itemize}
    \item Das IceCube ermöglicht es Supernovae schneller zu finden, 
    wodurch man mehr Daten über die Supernovae sammeln kann.

    \item Nachweis von magnetischen Monopolen wird durch IceCube ermöglicht

    \item Im Vergleich zu Antares stellt IceCube die wesentlich effektivere Nachweismethode dar

    \item Mit PINGU hätte man innerhalb einiger Jahre die Masse von Neutrinos bestimmen können.

    \item Ausgehend von den Resultaten von IceCube sollte man ähnlich Projekte mit Fördergelder subventionieren

    \item Wasser im gefrorenen Zustand stellt ein besseres Material zur Wechselwirkung dar als im flüssigen Zustand (bezogen auf IceCube und Antares)
    
    \item Das Super-Kamionkande ist aufwendiger zu bauen, dafür ist es auch eher in anderen Bereichen tätig.
    
    \item Das DUNE-Experiment ist deutlich aufwendiger und teurer als das IceCube, doch es wird vermutlich auch
    mehr wissenschaftlichen Output haben.

    \item Das IceCube-Projekt ist besonders durch seine Kollaboration erfolgreich, welche so groß geworden ist.
    
    \item In der Neutrinophysik stehen noch viele offene Fragen, es könnte das interessanteste Fachbereich der Physik
    sein in den nächsten Jahren. 

    \end{itemize}
    
