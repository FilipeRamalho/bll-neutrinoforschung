\chapter{Inwiefern hat das IceCube-Projekt seine Ziele erreicht} 
    \vspace{8pt}
    Das IceCube hat hoffensichtlich nicht so hohe Ziele wie andere Detektoren, dennoch erweist sich das
    IceCube als durchaus wichtig für die Astroteilchenphysik.
    Zudem ist das IceCube erst seit 2014 voll in Betrieb, also könnten durchaus noch wissentschaftliche Erfolge
    in Zukunft noch erreicht werden. \\
    Beim IceCube ist das größte Erfolg so eine große Kollaboration aufzubauen, die sich nur der Neutrinophysik widmet.
    Dies führte auch zu neuen Projekten. Auch hat erst diese Kooperation Projekte wie das DUNE-Experiment erlaubt.
    Die bereits jetzt errungenen wissentschaftlichen Erfolge sollte dennoch nicht kleinreden, zum einem wurde das Konzept 
    des IceCube als erfolgreich bewiesen, zum anderen konnte es den Neutrinoflux einer Supernova wahrnehmen.  
    Das IceCube bietet somit die Grundlage für die nächste Generation an Neutrino-Detektoren, wie den Hyper-Kamionkande, 
    den DUNE. \\
    Auch hat das IceCube einige hochenergetische Neutrinos gefunden, welche natürlich auch analysiert wurden.
    Insbesondere wurden kosmische $\nu_\mu$ entdeckt. 
    Zwar ist das IceCube nicht auf den gleichen Bekanntheitsniveau wie zum Beispiel das LHC in der Schweiz, hat es 
    dennoch wichtige Leistungen erbracht. Diese Leistungen sollen die Astroteilchenphysik stärken. In der wissentschaftlichen
    Gemeinde hat sich auch wieder ein "Hype" um Neutrinos entwickelt. In den nächsten Jahren wird sich nun zeigen müssen
    ob dieser "Hype" gerechtfertigt ist oder ob Neutrinos nur ganz gewöhnliche Elementarteilchen sind. Sie haben definitv 
    das Potenzial das Standardmodell zu erweitern ohne, dass man ganz neue Teilchen einführen muss.
    
