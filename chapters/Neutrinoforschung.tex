\chapter{Neutrinoforschung} 
    \vspace{8pt}
    \section{Das Neutrino}

    Das Neutrino ist ein subatomares Teilchen der Klasse der Leptonen ohne elektrische Ladung
        , es unterliegt somit der schwachen Wechselwirkung. 
    Nach dem Standardmodell ist das Neutrino ein punktförmiges Teilchen. Es gibt 3 Generationen von Neutrinos 
        mit jeweils anderer Masse. \\ \cite{Stoecker2000} 
    \begin{center}
        \begin{tabular}{ | l | c | } \hline
            Bezeichnung & Masse (MeV) \\ \hline
            Elektron-Neutrino & >7,3 $\cdot 10^{-6}$\\ 
            Muon-Neutrino & <0,27  \\ 
            Tau-Neutrino & <31 \\ \hline
        \end{tabular}             
    \end{center}
    Es gilt zu beachten, dass die Masse nicht genau angegeben ist, 
        da diese noch nicht genau bestimmt wurde, 
        doch es konnten bisher Obergrenzen bestimmt werden. \\
    Neutrinos können entweder natürliche Quellen haben wie kosmische, solare, atmosphärische oder Geoneutrinos
    Zudem gibt es von künstlichen Quellen auch noch Reaktorneutrinos und Beschleunigerneutrinos. \\
    Neutrinos könnten Anwendung finden in der Reaktorkontrolle bei der Überprüfung der Plutoniumproduktion, 
        indem man die Antineutrinoemissionen misst. \cite{Krauter2006}
    Insbesonders in der Astrophysik sind die Neutrinos von hoher Bedeutung. 
    Da sie nur schwach wechselwirken durchdringen sie fast jede Materie und so kann man mit ihnen Bereiche untersuchen die man mit anderer Strahlung nicht untersuchen kann.
    Zudem ist die Masse von Neutrinos bedeutend für viele astrophysikalische Theorien. \cite{Gelmini2010}
    \section{Geschichte} 
    
    \section{Aktuelle Forschung}
    \section{Zukünftige Forschung}
    \section{Forschung am IceCube}