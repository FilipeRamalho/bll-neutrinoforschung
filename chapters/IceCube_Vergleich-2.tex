    \section{Antares (Astronomy with a Neutrino Telescope and Abyss environmental Research)}

    Bei Antares handelt es sich ebenfalls um ein Neutrinoteleskop, welches Neutrinos kosmischer Herkunft 
    untersucht. Die Ziele, die beide Teleskope verfolgen sind nahezu identisch, wie die Ermittlung der 
    Strahlungsquellen kosmischer Höhenstrahlung. Ihre Funktionsweisen sind sich ebenfalls sehr identisch. 
    Es wird sich bei Antares zu Nutze gemacht, dass in seltenen Fällen die Neutrinos mit transparentem 
    Material wechselwirken. Antares befindet sich im Mittelmeer und man setzt darauf, dass die Neutrinos 
    mit dem Wasser wechselwirken. Ähnlich wie bei IceCube entstehen daraus dann Sekundärteilchen wie Myonen, 
    die dann mithilfe von Detektoren registriert werden. Beide Detektoren machen sich die seltene 
    Wechselwirkung von H2O mit Neutrinos zu nutze. Auch bei Antares entsteht Tscherenkowlicht. 
    Wie beim IceCube wurden auch hier einige Stränge mit optischen Modulen im Medium platziert. 
    Im Gegensatz zu IceCube wird es hier ermöglicht die Detektoren wieder zu erreichen und zu warten für den 
    Fall, dass sie ausfallen sollten. Antares stellt auch eher einen Prototyp für einen im Aufbau befindlichen 
    Teilchendetektor dar, nämlich KM3NeT. Dieser Detektor wird Antares ablösen, da Antares nur Messungen im 
    Bereich von 10 GeV bis 100 GeV wahrnehmen kann. Für höherenergetische Messungen ist das Teleskop zu 
    ungenau, was eigentlich essentiell ist, da höherenergetische Neutrinos eher Aufschluss darüber geben 
    woher Neutrinos stammen und welche Bedeutung sie haben. IceCube im Vergleich misst vor allem Daten im 
    Bereich von $\geq$200 GeV, was einige irrelevante Messungen wie irdische Neutrinos vernachlässigt. KM3Net 
    soll dann wesentlich größer werden mit 600 Strängen und 12000 Photomultiplern und ein größeres Spektrum 
    an möglich zu messbaren Energien. Beide Detektoren richten den Fokus auf Myonneutrinos, da sie nach 
    der Kernreaktion mit dem Medium ihre Bewegungsrichtung beibehalten. Sowohl IceCube als auch Antares 
    werden für andere Forschungen außerhalb der Neutrinoforschung eingesetzt. Durch die Positionierung 
    im Mittelmeer lässt es auch biologische Untersuchungsmöglichkeiten zu, wie die Erforschung der 
    Tierwelt oder von Biolumineszenz. IceCube bittet keine solcher zusätzlichen biologischen Ergänzungen, 
    da dies einfach nicht möglich ist durch die Positionierung im Eis. Jedoch findet IceCube auch Anwendung 
    bei der Untersuchung magnetischer Monopole und dient eher bei physikalischen Untersuchungen. \cite{AntOV13}

    \section{Double-Chooz-Experiment}

    Beim Double-Chooz-Experiment handelt es sich um ein Neutrionoteleskop, welches sich durch seine 
    \grqq Eigenschaften\grqq{} wesentlich mehr von IceCube unterscheidet als Antares. Dies beginnt schon bei der 
    Ausrichtung der Forschung. Während bei IceCube nach den Quellen von Neutrinos geforscht wird, wird 
    beim Double-Chooz-Experiment die Neutrinooszillation untersucht. Darunter versteht man die Eigenschaft 
    des Neutrinos sich in eine andere Art des Neutrinos umzuwandeln, wie beispielsweise von einem Antineutrino 
    in ein Myonneutrino. Dieses Experiment wird im Kernkraftwerk Chooz betrieben. Das Ziel besteht darin 
    herauszufinden wie hoch die Wahrscheinlichkeit ist, dass Neutrinooszillation stattfindet, dafür werden 
    zwei Detektoren, in unterschiedlich weitem Abstand vom Reaktor, positioniert. Als Nebenprodukt des 
    inversen radioaktiven $\beta$-Zerfalls entstehen Antineutrinos, die sich willkürlich in alle Richtungen ausbreiten:
    \begin{center}
        $\overline{\nu}_e + p \rightarrow n + e^+ $
    \end{center}
    Man sieht, dass ein Antielektronneutrino und ein Proton in ein Neutron und in ein Positron umwandelt. 
    Die beiden Detektoren registrieren lediglich Antineutrinos. Wenn also der weit entfernte Detektor 
    weniger Messergebnisse hat, kann man daraus schließen, dass sich die Antineutrinos eher umgewandelt 
    haben als im nahen Detektor. Dazu erzeugt das Neutron Szintillationslicht. Dies geschieht indem das 
    Neutron auf ein Elektron des Elements Gadolinium trifft, welches sich in dem Detektor befindet. 
    Das Elektron wird in einen energetisch höheren Zustand gehoben und fällt wieder in seine ursprüngliche 
    Bahn, wodurch Energie in Form von Licht frei wird. Wenn man die Messdaten der beiden Detektoren miteinander 
    vergleicht, lässt sich die Umwandlungswahrscheinlichkeit annäherungsweise bestimmen. Anhand dieser 
    Forschungsart lässt sich erkennen, dass die Ziele der einzelnen Forschungsstätte sich stark unterscheiden, 
    aber trotzdem alle Ergebnisse dazu beitragen die Neutrinos besser zu verstehen. \cite{DCCos12}

    \section{Gallex (Gallium-Experiment)}

    Gallex beschäftigte sich überwiegend mit solaren Neutrinos (also von der Sonne emittierte Neutrinos). Anders als IceCube ist Gallex nicht mehr in Betrieb, 
    da die Ziele der Forschung erreicht wurden. Mit diesem Detektor sollte der Beweis für die Existenz solarer Neutrinos erbracht werden. 
    Mit dem Beweis sollten Theorien zur Energieerzeugung der Sonne bewiesen oder widerlegt werden. Wie der Name schon andeutet handelt 
    es sich um einen Nachweis mit dem Element Gallium. Der Detektor war gefüllt mit einer großen Menge Galliumtrichlorid-Lösung, 
    die zu etwa 30\% aus Gallium besteht. Beim Auftreffen eines Neutrinos gibt es einen sogenannten inversen $\beta$-Zerfall, also eine
    Kernreaktion. Dort entsteht kein Neutrino sondern der Zerfall wird durch ein Neutrino ausgelöst:
    \begin{center}
        $\nu_e+^{71}Ga\rightarrow e^- + ^{71}Ge^+$
    \end{center}
    Aus dem Neutrino und dem Gallium entstehen ein Elektron und Germanium. Gallium wird verwendet, 
    da es eine niedrige Schwellenenergie besitzt, was zur Folge hat, dass auch Neutrinos mit niedrigen 
    Energien einen inversen $\beta$-Zerfall auslösen können. Dies ist sinnvoll, da die solaren Neutrinos eine 
    eher geringe Energie haben. Einige solare Neutrinos erreichen nicht die Erde oder durchdringen deshalb 
    nicht die Erde aufgrund ihrer geringen Energie. Bei Gallex beträgt diese Schwellenenergie etwa 233 keV 
    und bei IceCube liegt diese bei ungefähr 200 GeV. Das ist in etwa ein Unterschied vom Faktor $10^6$. Dies 
    zeigt, dass die solaren Neutrinos eine deutlich geringere Energie aufweisen. Das entstandene Germanium 
    wurde dann extrahiert und in das Gas Monogerman umgewandelt. Dieses hat eine recht kurze Halbwertszeit 
    (~11,4 Tage), mit der man nach jedem Zerfall ein Neutrino \grqq eingefangen\grqq{} hatte. Durch diesen Detektor konnte 
    man den ersten Nachweis dafür erbringen, dass Neutrinos oszillieren, denn mathematische Modelle sagten 
    mehr Registrierungen von Neutrinos hervor. Allerdings registriert dieser Detektor lediglich 
    Elektronenneutrinos, was bedeutet, dass die Neutrinos oszilliert haben. Durch die Oszillation wurden die 
    Elektronenneutrinos in andere Neutrinos umgewandelt, die der Detektor nicht registriert. Eine Bedingung 
    für die Oszillation ist, dass die Neutrinos Masse haben, was dadurch bewiesen wurde. Vorher nahm man an, 
    dass Neutrinos masselos sind, was durch diese Experimente widerlegt wurde. IceCube weißt ebenfalls große 
    Erfolge auf, wie die Entdeckung von Neutrinoquellen. Allerdings legen solche Entdeckungen den Grundbaustein 
    für nachfolgende Projekte wie IceCube. \cite{PhyGA13}

    \section{Ergebnisse des Vergleichs}

    Abschließend, nach einem ausführlichen Vergleich, stellt sich die Frage, ob es sinnvoll ist alle 
    Detektoren weiter zu betrieben oder ob nur bestimmte weiter betrieben und mit weiteren Fördermitteln 
    subventioniert werden sollen. Zuerst sollte man beachten, dass alle genannten Observatorien wichtige 
    Daten gesammelt haben, die dazu beitragen die Neutrinos besser zu verstehen. Zum anderen ist es sinnvoll 
    möglichst viele Daten zu sammeln, da die Wechselwirkung zwischen kosmischen Neutrino und Materie selten 
    stattfindet. Deshalb ist es nützlich möglichst viele unterschiedliche Nachweismedien zu verwenden, um zu 
    testen, welches das mit der häufigsten Interaktionsquote ist, um die zukünftig Forschung zu verbessern. 
    Nicht  zu  vergessen  ist,  dass  sich  die  Ziele,  die  man  erreichen  und  die Grundlagen, die man 
    versucht zu erklären unterschiedlich sind. Die Forschung richtet sich zwar nach den Neutrinos, aber 
    spezifisch beschäftigen ich die Kollaborationen mit unterschiedlichen Schwerpunkten und Kollaborationen 
    mit den selben Schwerpunkten haben die Möglichkeit einer engen Kooperation, um ihre Ergebnisse zu 
    vergleichen und effizienter zu arbeiten und zu forschen. Außerdem muss man bedenken, dass viele Fragen 
    in einem geringen Zeitraum beantwortet werden können und nicht alle in einem großen zeitlichen Abstand 
    beantwortet werden müssen. Nicht zu vergessen ist außerdem, dass bestimmte Daten anderen 
    Forschungseinrichtungen dabei helfen können ihre Forschung zu optimieren und anzupassen. Deshalb ist es 
    logisch, dass mehrere Forschungsprojekte zur selben Zeit laufen. Selbstverständlich müssen diese 
    Forschungseinrichtungen irgendwann außer Betrieb gesetzt werden, da diese sonst nur weitere Gelder 
    beanspruchen würden, die an anderen Stellen sinnvoller sein könnten. Dies sieht man beispielsweise 
    gut an Gallex. Gallex hat die aufgeworfenen Fragen beantwortet. Eine Fortführung wäre überflüssig, 
    da sie keine weiteren Antworten mehr liefern könnte. Anders verhält es sich beispielsweise bei IceCube. 
    Während man mit Gallex einige Theorien klären sollte, versucht man mit IceCube auch Quellen zu 
    identifizieren, die uns weiter helfen könnten und es gibt sehr viele Quellen, die man noch lokalisieren 
    könnte. Außerdem geht IceCube nun weit über seine eigentlichen Ziele hinaus. Mit IceCube gibt es noch 
    viele Forschungsmöglichkeiten. Als Beispiel kann man anführen, dass jetzt nach den GZK-Neutrinos 
    geforscht werden soll. Das Prinzip von IceCube ermöglicht theoretisch die Registrierung dieser Neutrinos. 
    Die Wahrscheinlichkeit, dass diese Neutrinos mit dem Eis wechselwirken ist sehr gering. Man muss bedenken, 
    wie viele Millionen Neutrinos uns pro Sekunde durchqueren. Der prozentuale Anteil davon wie viele davon mit 
    Materie wechselwirken ist gering. Nun muss man bedenken, dass es vermutlich mehr normale kosmische 
    Neutrinos gibt als GZK-Neutrinos, was es ebenfalls wieder unwahrscheinlicher macht, dass sie mit 
    Materie wechselwirken. Deshalb ist es umso wichtiger mehr Daten zu sammeln, was vor allem mit IceCube 
    möglich ist, da IceCube das größte Volumen der Neutrinoteilchendetektoren darstellt. Wenn sich nun die 
    Frage stellt, ob man nun Antares oder IceCube neue Fördermittel zur Verfügung stellt, wäre es natürlich 
    wünschenswert, das beide mit neuen Gelder ausgestattet werden, denn beide würden die Neutrinoforschung 
    voranbringen. Wenn dies jedoch nicht möglich wäre und eine Entscheidung gefällt werden muss, dann würde 
    ich es am sinnvollsten erachten, dass IceCube eher mit neunen Fördergeldern unterstützt wird. Dies halte 
    ich zum einen für sinnvoll, da IceCube schon gebaut wurde und Antares müsste noch gebaut werden und ist 
    dann in etwa vergleichbar mit IceCube, aber ohne die Erweiterung, denn im Prinzip sind es die selben 
    Detektoren, welche sich lediglich beide in$H_2O$ befinden, nur in einem anderen Aggregatzustand. Außerdem 
    hat die IceCube-Kollaboration deutlich mehr Erfahrung in der Neutrinoforschung und weiß wie man am besten
    vorgeht und sind schon eher in dem Denken drin, wie man das Projekt noch optimieren könnte oder wie man die 
    Daten am besten analysiert. Nicht zu vergessen ist, dass IceCube schon konkrete Pläne hat, wie man den 
    Detektor am besten ausbaut um auch in anderen Gebieten der Neutrinos zu forschen, hier lässt sich auf die 
    PINGU-Erweiterung verweisen, die zwar nicht gebaut wurde, aber immer noch geplant wird, wenn Fördergelder 
    bereitgestellt werden. \cite{Neutrino14}

