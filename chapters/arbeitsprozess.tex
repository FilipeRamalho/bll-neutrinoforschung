\chapter*{Arbeitprozess} 

    \vspace{8pt}

    Unser Physik-Fachlehrer empfahl uns den Besuch einer IceCube-Masterclass 
    an der Johannes-Gutenberg-Universität im letzten Jahr. Dort wurden wir in
    die Neutrinophysik eingeführt, zudem wurde uns das IceCube-Projekt vorgestellt.
    Zum Ende des Tages durften wir selber auch tätig werden, wir analysierten Events aus
    dem Detektor und durften auch selber nachforschen was dort gemacht wird.
    Dies fasznierte uns so sehr, dass wir uns dazu entschieden gemeinsam eine Arbeit über das
    Projekt zu schreiben. Wir wollten unser Wissen erweitern. Mit den nächsten
    Erweiterungen des Neutrino-Detektors glauben wir, dass eine Erweiterungen des momentanen
    Verständnis der Physik durch diese Detektoren möglich ist. \\
    Zunächst informierten wir uns grob über das Projekt und fingen an mehr über die Neutrinophysik
    zu lernen. Daraufhin haben wir eine Übersicht formuliert über die Themen, über welche wir etwas
    schreiben wollten und schreiben mussten. Da dies in Zusammenarbeit geschah teilten wir uns
    die Themen auf und jeder informierte sich genauer. 
    Während unserer Arbeit stellten wir fest, dass einige Institute nicht viele Spezifikationen 
    veröffentlichten, dies erschwerte ein Vergleich an manchen Stellen. 
    In unseren Gesprächen haben wir gelernt wie man fachlich und wissenschaftlich korrekt zitiert und 
    Arbeiten schreibt. 

