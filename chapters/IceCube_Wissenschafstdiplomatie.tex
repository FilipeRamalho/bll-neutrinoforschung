\chapter{Rolle des IceCube-Projekts in der Wissenschaftsdiplomatie} 
    \vspace{8pt}

    \section{Finanzierung und Kooperationen}
    
    Die Kosten des Projekts beliefen sich auf etwa 279 Millionen US-Dollar. 
    Die Kosten für das Projekt wurden hauptsächlich von der \grqq National Science Foundation\grqq{} getragen. 
    Das sind etwa 242 Millionen US-Dollar, die von der \grqq National Science Foundation\grqq{}  übernommen wurden. 
    Die weitere Finanzierung erfolgte durch Universitäten und Institute aus vielen verschiedenen Ländern 
    und einigen Forschern, neben den USA waren auch Länder wie Deutschland, die Niederlande und Belgien 
    bei der Finanzierung vertreten. Das Bundesministerium für \grqq Bildung und Forschung\grqq{} und die 
    \grqq Deutsche Forschungsgemeinschaft\grqq{} (DFG) subventionierten die Konstruktion des Observatoriums. 
    Aber auch beispielsweise das \grqq Institut zur Förderung von Innovation durch Wissenschaft und 
    Technologie in Flandern\grqq{}, welches sich in Belgien befindet, leistete finanzielle Unterstützung.
    Deutschland nimmt eine wichtige Rolle beim IceCube-Projekt ein. Des Weiteren ist DESY für die Prüfung 
    von 1250 optischen Modulen tätig gewesen oder auch für die Datenanalyse lieferten sie wichtige Beiträge. 
    Beispielsweise waren sie dafür zuständig die Software zu schreiben und elektrische Komponenten zu 
    entwickeln. Nicht zu vergessen ist, dass DESY stets an neuen Methoden für den Nachweis von Teilchen 
    arbeitet. Die \grqq National Science Foundation\grqq{} hielt es für angebracht der Universität von 
    Wisconsin-Madison die Führung zu übertragen, da diese Universität bereits mit dem Projekt AMANDA 
    beauftragt gewesen ist und die Ergebnisse zufriedenstellend waren. Sie hatten somit Erfahrung 
    im Bereich der Forschung von (kosmischer) Höhenstrahlung \cite{FAQ13}.

    \section{IceCube-Kollaboration}

    Etwa 300 Physiker bilden die IceCube-Kollaboration, darunter befinden sich Physiker, 
    Informatiker, Ingenieure etc. Seit November 2017 besteht sie aus 50 Einrichtungen in 12 Ländern, 
    welche an dem Detektor arbeiten, in dem sie ihn beispielsweise auf Funktionalität der Sensoren 
    überprüfen und die Messdaten auswerten, aber viele Mitglieder waren auch der Konstruktion des Detektors 
    beteiligt. Die Aufgabenbereiche sind unzählig, einige werden direkt an der Südpolststion ausgeführt 
    und andere forschen von ihren Universitäten aus, da IceCube nicht die einzige Aufgabe der 
    IceCube-Kollaboration darstellt. Ein Beispiel für ein deutsches Forschungsinstitut ist das 
    \grqq Deutsche Elektronensynchrotron\grqq{} (DESY), aber auch viele deutsche Universitäten wie die Technische 
    Universität München, die Ruhr-Universität Bochum oder auch die Johannes Gutenberg Universität sind an 
    diesem Projekt beteiligt und ebenso ist auch die Humboldt-Universität zu Berlin eine maßgeblich beteiligte 
    Universität \cite{DeFor13} \cite{FAQ13}.

    \section{Wissenschaftliche Bedeutung}

    Dieses Projekt ermöglicht eine Zusammenarbeit von vielen Universitäten und Instituten, 
    die eng miteinander agieren, um eine intensive und effiziente Forschung zu gewährleisten. 
    Es ist sinnvoll, dass möglichst viele Wissenschaftler zusammenarbeiten, da sich so viele Menschen mit 
    unterschiedlichen Arbeitsmethoden und Denkweisen aufeinandertreffen. Es ist möglich, dass diese voneinander 
    lernen können und zusammen den idealsten Weg finden, um Probleme zu lösen und die Forschung zu verbessern. 
    Außerdem stellt eine wissenschaftliche Zusammenarbeit auch eine Brücke zwischen zwei Gesellschaften dar, 
    um gemeinsame Strategien zur Überwindung globaler Differenzen und Problematiken zu konstruieren. 
    Nicht unerwähnt sollte die Tatsache bleiben, dass einige Forschungsstationen auch zusammenarbeiten und 
    nicht ausschließlich Konkurrenzdenken betreiben. Ein Beispiel dafür ist die Kooperation von IceCube mit 
    HAWC (ebenfalls ein auf Tscherenkow-Strahlung basierendes Neutrinoteleskop). Die verschiedenen 
    Kollaborationen haben Einfallsrichtungen der kosmischen Strahlung bei gleicher Energie in 
    unterschiedlichen Himmelrichtungen abgeglichen. Das Ziel bestand darin die Ausbreitung der 
    kosmischen Strahlung, mit einer geringeren Energie von etwa 10TeV, zu untersuchen, um die 
    Anisotropie besser zu verstehen. Das findet natürlich nicht ausschließlich bei IceCube statt, 
    aber IceCube ist ein passendes Beispiel dafür, dass internationale Zusammenarbeit, bei IceCube 
    sind Wissenschaftler aus zwölf Ländern vertreten, einige Erfolge bringt. Dies lässt sich vor allem 
    durch die wissenschaftlichen Leistungen festmachen, die durch IceCube geleistet wurden \cite{DeFor13} \cite{IceHa18}.

