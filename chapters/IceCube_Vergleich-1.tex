\chapter{Vergleich zu anderen Forschungsstätten} 
\vspace{8pt}

\section{Super-Kamiokande}

Der Super-Kamiokande steht in der japanischen Stadt Hida, es handelt sich bei diesen um einen
Cherenkov-Detektor. Er ist der Nachfolger des Kamiokande, welches in seiner letzen Ausführung,
des Kamiokande-III, 1995 endgültig außer Betrieb ging.
Einer der Hauptmotivationen des Kamiokande-Projekts war es den Protonzerfall näher zu untersuchen. 
Da das originale Kamiokande in diesen Bereich keine Erfolge lieferte entschied man sich für 
den Bau des Super-Kamiokande. \cite{Fukanda2003}

\subsection{Geschichte}

Die Datenaufnahme wurde beim Super-Kamiokande im April 1996 begonnen. Es durchging immer wieder 
Reparaturen und wurde schon mehrmals nachgerüstet. \cite{Fukanda2003} Im November 2001 
geschah ein Unfall, ungefähr 7000 PMTs (optische Sensoren)implodierten. Es wurde vermutet, 
dass die Implosion einer PMT eine Schockwelle im Wasser verursachte, welche die andere PMT's
implodieren ließ. Der Super-Kamiokande musste schnell wiederaufgebaut werden, weshalb man sich 
entschied nicht alle PMTs zu reparieren, sondern man beließ es bei der Hälfte, so konnte der 
Betrieb schnell wieder aufgenommen werden. \cite{Cartlidge2001}

\subsubsection{Neutrinooszillationen}

Neutrinooszillationen sind ein interessantes theoretisches Konzept, experimentell wurden sie 
aber 1998 zum ersten Mal vom Super-Kamiokande nachgewiesen. 
Bei der Wechselwirkung zwischen kosmischer Strahlung und Atomkernen in der Atmosphäre 
entstehen atmosphärische Neutrinos als Zerfallprodukt. Dort dominieren zwei Reaktionen:
\begin {center}
$\tau \rightarrow \mu + \nu_\mu$\\
$\mu \rightarrow e + \overline{\nu}_\mu + \nu_e$
\end {center}
Also müsste sich logischerweise ein Verhältnis von 2$\nu_mu/\overline{\nu}\mu$:1$\nu_e/\overline{\nu}_e$ 
ergeben. Beim Super-Kamiokande wurde diese Annahme überprüft, man kann aber auf ein Verhältnis 
von 1:1. Dieser Unterschied konnte nicht durch statische Unsicherheiten oder experimentelle 
Unsicherheiten erklärt werden. Wenn man aber eine Neutrinooszillation von 
$\nu_\mu \rightarrow \nu_tau/\nu_?$ mitbedenkt, dann würde sich dieses Verhältnis erklären lassen. 
Also könnte dieser Effekt durch Neutrinooszillationen erklärt werden. \cite{FUKUDA2003} MacDonald und Takaaki 
Kajita haben 2015 ein Nobelpreis bekommen für den deren Arbeit an Neutrinooszillation, wobei
Kajita hauptsächlich am Super-Kamiokande gerabeitet hat und MacDonald an der SNO-Kollaboration.

\subsubsection{Weitere Experimente}

Um diese Entdeckung zu untermauern wurde anschließend das K2K-Experiment durchgeführt. Bei diesem Experiment
wurden $\nu_\mu$ vom KEK über eine 250km lange Strecke gesendet, das Endziel war der Super-Kamiokande. 
Das Super-Kamiokande konnte weiterhin kein $nu_\tau/\overline{\nu}_\tau$ registrieren. \cite{Collaboration2002}
Wenn die $nu_\mu/\overline{\nu}_\mu$ nicht oszillieren dann würde man $158,1^{+9,2}_{-8,6}$ gemessene Ereignisse 
am Super-Kamiokande erwarten. Tatsächlich wurden aber 112 gemessen. Die Wahrscheinlichkeit, dass die 
ohne Neutrinooszillation passiert ist liegt bei 0,0015\%. Man kann also mit hoher Sicherheit annehmen, 
dass die Resultate durch Neutrinooszillationen erklärt werden. \cite{Ahn2006}\\
Mit dem Super-Kamiokande wurde mit K2K und der Untersuchung von atmosphärischen Neutrinos nur die Oszillation
zwischen $\nu_\mu - \nu_\tau$. Wenn auch seltener gibt es auch eine Oszillation zwischen $\nu_\mu$ und 
$\nu_e$. Das T2K sollte das untersuchen. Beim J-PARC wurden wie beim vorherigen Experiment $v_\mu$ über 
eine 295km lange Strecke gesendet. Die Intesität wurde beim T2K-Experiment erhöht um 2 Magnituden.\cite{Yuichi2006}
Aktuell läuft das T2K-Experiment immer noch, also gibt es noch kein entsprechendes Abschlussbericht, doch
es wurden zwei Haupterfolge gemeldet. \\
Zum einen war der eine Erfolg der Beweis, dass $\nu_\mu$ sich tatsächlich in $\nu_e$ wandlen können. Mit diesem Erfolg
hat sich auch der Fokus des T2K-Experiment auf eine Untersuchung von CP-Verletzungen. \cite{T2K2017}
Es wurde auch eine mögliche CP-Verletzung gefunden, da man ein Unterschied zwischen der Oszillation von Neutrinos
und Antineutrinos entdeckt. Die Ereignisanzahl ist zu gering, weshalb man noch bis 2026 weiter Daten sammlen muss
um dies mit einer höheren Konfidenz zu bestätigen. \cite{T2K2013} 

\subsection{Hyper-Kamiokande}

Die PINGU-Erweiterung beim IceCube-Projekt scheiterte, währenddessen wurde die Erweiterung für das Kamiokande genehmigt. 
Diese Erweiterung wird ein deutlich größeren Cherenkov-Detektor benutzen, auch die PMTs wurden verbessert. Das Tankvolumen
steigert sich um das 20fache im Vergleich zum Super-Kamiokande. Es sollen auch 2 Tänke gebaut werden, wobei der erste
Tank in Betrieb gehen soll bevor der zweite Tank fertig gebaut wird.
Auch hier sollen $\nu$ mithilfe des J-PARC über eine lange Strecke gesendet. Also handelt es sich um die nächste
Generation des T2K-Experiments. Mit dem neuen Detektor wird man auch wieder atmosphärische Neutrinos und astrophysikalische
Neutrinos untersuchen. 
Mit dem neuen Detektor bietet sich eine erneute Untersuchung, da durch die
erhöhte Sensivität mehr Ereignisse messen kann und man kann Parameter genauer bestimmen.
Der zweite Tank des Hyper-Kamiokande soll nach Plan in Südkorea gebaut werden. Es hat sich herausgestellt, dass
der von J-Parc ausgesendete Neutrinostrahl in Südkorea auftrifft und somit wäre es möglich ein
Detektor dort zu bauen, welches Neutrinos, welche eine noch größere Strecke hinter sich haben,
messen würde. Damit kann man Neutrinooszillationen noch besser untersuchen.  
Es dürfte ungefähr 10 Jahre brauchen bis der erste Tank in Hida fertig gebaut wird, nachdem dieser läuft 
soll der Bau des zweiten Tanks in Sükorea nur noch 6 Jahre brauchen. Ob es tatsächlich in Korea gebaut werden
soll wird momemtan untersucht. \cite{Lodovico2017}
Der Bau des Hyper-Kamiokande soll 2020 beginnen. \cite{HyperK2018}

\subsection{Vergleich}

\subsubsection{Konstruktion}

Sowohl der IceCube-Detektor als auch der Kamiokande benutzen Cherenkov-Strahlung um Neutrinos zu messen.
Beim Kamiokande ist der Bau deutlich aufwendiger wie man am Hyper-Kamiokande sehen kann, man muss geologische Untersuchungen 
machen, man muss eine entsprechende Stelle aushöhlen und einen Tunnel zu der Stelle graben, dann noch ein Tank bauen, 
Wasser purifizieren und es in den Tank füllen. \cite{Lodovico2017} 
Beim IceCube-Detektor musste man stattdessen tiefe Löcher ins Eis schmelzen. Man braucht 2 Tage um das Loch zu schmelzen
und 11 Stunden um den optischen Sensor anzubringen. Problematisch ist der Transport von Materialien von den entsprechenden
Produktionsstätten zum Südpol.  \\
2001 begann die Planung für den IceCube-Detektor. Innerhalb von 2 Monaten nach Baubeginn im Jahre 2004, war der IceCube-Detektor 
betriebsbereits, aber es hat 6 Jahre gebraucht, damit es tatsächlich in seiner vollen Größe fertig war. 
Das Super-Kamiokande begann mit der Planung 1991 und wurde 1996 vollendet.
Zwar war die gesamte Bauperiode beim IceCube gering, da man es aber über die Jahre verteilte, also man hat diesen in Phasen
gebaut, dauerte es länger als beim Super-Kamiokande.  \\
Auch waren die Baukosten entsprechend höher. Beim Super-Kamiokande wurden 100M\$ vorgesehen und der IceCube hat tatsächlich 
279M\$ gekostet. 
Die Konstruktion des IceCube-Detektors war zwar einfacher, aber auch teurer. 

\subsubsection{Wissenschaftlicher Fokus}

Die größten Erfolge des Super-Kamiokande liegen im Bereich der Neutrinooszillation, beim IceCube war es hingegen die 
Etabilierung eines Supernovae-Frühwarnsystem und die erfolgreiche Erprobung am Blazar TXS 0506+056,die bisher größte astrophysikalische 
Neutrino-Quelle die gefunden wurde, einer deren großen Erfolge. \\
Man erkennt ein klaren Unterschied an den thematischen Fokus der beiden Detektoren. Während das Super-Kamiokande sich der 
Untersuchung der Eigenschaften von Neutrinos widmet, ist der IceCube-Detektor mehr auf die astrophysikalische Herkunft von
Neutrinos fokussiert und die Untersuchung der Eigenschaften von Neutrinos im astrophysikalischen Zusammenhang. 
Durch die besondere Lage macht das IceCube-Projekt auch glaziologische Untersuchungen, das Super-Kamiokande hat keiner
interdisziplinäre Untersuchungen. \\
Das Super-Kamiokande hat deutliche Fortschritte in der Teilchenphysik geleistet, das IceCube konnte in der Astrophysik
aber bereits beweisen, dass es in seinem Fachbereich was leisten kann. Auch die wichtigen Leistungen des Super-Kamiokande kamen 
nicht über Nacht sondern aber auch viele Jahre Forschung gebraucht. Das IceCube musste bevor es anfangen kann wichtige Beiträge
zur Astrophysik zu leisten noch beweisen, dass sein eigentliches Konzept auch viabel ist. \\
Es hat dennoch bereits viele Daten über Neutrinos gesammelt und in Kooperation mit anderen Instituten auch kleinere Fortschritte
in der Neutrinophysik geleistet. 

\section{DUNE}

Das DUNE ist ein in Planung befindendes Neutrinoexperiment. Es besteht aus 2 Neutrinodetektoren und einen Protonenbeschleuniger 
für die Produktion von Neutrinos. Hinter dem DUNE-Experiment steht eine große Kollaboration, die drei wichtigsten Institute
sind das CERN in der Schweiz, das Fermilab in Illnois und das Sanford Lab in South Dakota.  

\subsubsection{Aufbau}

Das DUNE-Experiment ist in Near-Site-Facilities(NSFs) und Far-Site-Facilities(FSFs) unterteilt, wobei die NSFs in der Nähe des 
Protonenbeschleunigers sind und die FSFs 1300 Kilometer vom Protonenbeschleuniger entfernt.
Zu den NSFs gehören der Fermilab mit seine Protonenbeschleuniger und der anschließenden Neutrinoproduktion, auch
gibt es eine Neutrinodetektor direkt am Fermilab. Zu den FSFs gehört ein weiterer Detektor in der Sanford Underground
Research Facility (Siehe \ref{fig:dune})

\subsubsection{Vorgeschichte}

Zunächst sollte das DUNE-Experiment in us-amerikanischer Einzelarbeit gebaut werden. Es hieß damals auch noch LBNE. \cite{Moore2015}
LBNE wurde dan aufgelöst und durch das DUNE-Experiment abgelöst, man wollte mit anderen Instituten zusammenarbeiten. 
Das LBNE sollte näher an der Oberfläche sein als das DUNE-Experiment. 2015 wurde das LBNE durch das ELBNF abgelöst und später auch
in das Dune-Experiment umbenannt. Der Bau begann 2017, 2018 konnten schon erste Messungen geführt werden. 2026 soll der Bau abgeschlossen
werden. \cite{Fermilab2015}

\subsection{Untersuchungsthemen}

Dieses Experiment soll in vielen Bereichen der Physik agieren. Man will astrophysikalische Untersuchungen machen zu Supernovae,
Untersuchungen in der Teilchenphysik zu der Grand Unified Theory (Vereinigung aller Grundkräfte inkl. Gravitation) und 
zu CP-Verletzungen. Auch soll die Masse des Neutrinos weiter untersucht werden. \cite{Brailsford2018}

\subsubsection{Supernovae}

Das DUNE-Experiment soll im Vergleich zu anderen Neutrino-Detektoren viel besser niederenergtische Teilchen messen. Damit
kann man die in einem Event freigesetzte Energie deutlich genauer bestimmen. Bei Supernovae werden zu deren Beginn eher $\nu_e$
emittiert und in der späteren Phase $\overline{\nu}_e$. Das DUNE-Experiment kann besser $\nu_e$ messen als $\overline{\nu}_e$ und
wird sich deshalb dafür eignen eher die frühe Phase von Supernovae zu untersuchen. Insbesondere erhofft man sich, dass mithilfe
von Daten über die Neutrinofluxe von anderen Detektoren, wie den Hyper-Kamiokande, man die Emission von Neutrinos bei der Supernovae
besser versteht. \cite{Ankowsi2016}

\subsubsection{Grand Unified Theory}

Nach einigen GUTs sollten Protonen zerfallen. Wie zum Beispiel nach dieser Reaktion:
\begin{center}
$ p \rightarrow K^+ + \overline{\nu} $
\end{center}
Einer der Zerfallsprodukte wäre ein Antineutrino. Das kann vom DUNE-Experiment gemessen werden.
Das K-Meson würde nach 12,8ns weitere zerfallen: 
\begin{center}
    $ K^+ \rightarrow \mu^+ + \nu_\mu $ /  $ K^+ \rightarrow \tau^+ + \tau_0 $
\end{center}

Nach theoretischen Überlegungen lässt sich auch bestimmen wie das entsprechende Spektrum aussieht. 
So müsste man zwei Peaks bei 257 MeV und 459 MeV sehen. Beim LENA-Projekt könnte man dann im
Laufe von 10 Jahren ungefähr 10 solcher Ereignisse sehen, sollten Protonen zerfallen, beim
DUNE gibt es noch keine genaue Angaben wie sensibel ist bei Protonenzerfällen. Da sie, die Messung
eines solchen deren Ziel ist, sollten das DUNE-Experiment entsprechend sensible Detektoren bauen. \cite{Undagoitia2008}

\subsubsection{CP-Verletzungen}

Wie beim Super-Kamiokande will man beim DUNE-Experiment CP-Verletzungen anhand von 
Neutrinooszillation nachweisen. 
Auch hier soll die Antineutrinooszillation mit der Neutrinooszillation verglichen werden. 
Das DUNE hat aber den Vorteil, dass es eine wesentlich längere Streck zwischen den Detektor und der Neutrinoproduktion
hat im Vergleich zum Super-Kamionkande. Nach 14 Jahre Laufzeit werden die vom DUNE gemessenen oder nicht gemessenen
Unterschied mit einer $3-\sigma$-Signifikanz stimmen. Sollte keine Effekt gemessen werden, dann wird das
DUNE-Experiment aussagen können wie groß die CP-Verletzung noch sein darf. 
\cite{Acciarri2016}

\subsubsection{Weitere Themen}

Das sind die vom Projekt selbst vorgenommenen Ziele, doch der große Umfang des Projekts und der Kollaboration führen dazu, dass
es noch weitere Überlegungen gibt das DUNE-Experiment in noch mehr Bereichen zu nutzen. 
Eine dieser Überlegungen ist mit dem DUNE nach sterilen Neutrinos zu suchen. Diese, wie zuvor erklärt, interagieren nur
durch Gravitation, das mach deren Messung besonders schwer. Beim DUNE sollen auch die Parameter der Neutrinooszillation
genauer gemessen werden, sollte diese nicht den theoretisch progonsizierten Werten, deutet das daraufhin, dass
das theoretische Verständnis des Neutrinos möglicherweise überarbeitet werden muss, dazu könnte eine Erweiterung
des Standardmodell um sterile Neutrinos gehören.\cite{Berryman2015}
Eine Erweiterung der Neutrinophysik könnten auch nicht standardmäßige Neutrinodetektoren bieten. Die Sensivität des DUNE sollte
ausreichen um zu bestimmen, ob es diese nicht standardmäßige Neutrinointeraktionen überhaupt gibt, es könnte
sogar reichen um die neuen Parameter zu bestimmen. Wenn die Datenlage gut ist, könnte man auch die Interaktion
näher bestimmen und gegebenfalls von sterilen Neutrinos unterschieden werden. Es wird erwartet, dass man
dennoch mehr Daten brauchen wird, als das DUNE liefern wird. \cite{Gouvea2016}

\subsection{Vergleich}

\subsubsection{Konstruktion}

Wie auch schon beim Super-Kamiokande ist der Bau deutlich aufwendiger. Zudem wird das DUNE-Experiment deutlich teurer
werden, als das IceCube-Projekt. Es wird nach jetztigen Schätzunden alleine der us-amerikanischen DOE 1,3-1,9 Billionen
Dollar kosten. 
Die Planungsphase für das DUNE-Experiment begann 2015 und der Bau soll 2026 fertig sein, das liegt ungefähr im gleichen
Rahmen wie beim IceCube, wobei das IceCube innerhalb von 3 Jahren in Betrieb ging, doch das DUNE-Experiment kann seinen Betrieb
erst nach Bauende beginnen. Hier zeigte sich auch wieder, dass die phasenweise Bauart des IceCubes den Vorteil hat, dass durch
diese es möglich ist relativ schnell mit Messungen zu beginnen.

\subsubsection{Wissenschaftlicher Fokus}

Das DUNE-Experiment hat sich ganz offensichtlich höhere Ziele gesetzt als das IceCube-Projekt. Anders als beim
Super-Kamionkande ist die Überscheidungen zwischen den beiden Detektoren größer. Sowohl das IceCube-Projekt als
auch das DUNE betreiben astrophysikalische Forschung im Bereich der Supernovae. Auch untersuchen beide die Masse
von Neutrinos, wobei fast alle Neutrino-Detektoren das untersuchen, dies eine sehr grundlegende Fragestellung ist, 
welche immernoch nicht genau beantwortet wurde. \\
Bei der Untersuchung von Supernovae könnten sich die beiden eine Rolle spielen, das IceCube hat bewiesen, dass es die
Neutrinofluxe aus Supernovae wahrnehmen kann und rechtzeitig andere Institute informiert über den Fund. So könnte
das IceCube ein Neutrinoflux einer Supernovae wahrnehmen und das DUNE-Experiment kann dann versuchen die niederenergtischen
Neutrinos zu messen, wobei es sein kann, dass keine mehr gemessen werden können, da wie zuvor festgestellt werden die niederenergtischen
Neutrinos zu Beginn der Supernova ausgestrahlt. \\
So eine zusammenarbeit könnte, aber auch ohne Warnsystem funktionieren. Beide Detektoren sollten den Neutrinoflux einer
Supernova wahrnehmen, beim IceCube wurde das bereits gemacht, auch spricht beim Aufbau der DUNE-Detektoren nichts dagegen, dass
sie wie das Super-Kamiokande Neutrinofluxe aus Supernovae wahrnehmen können. Man kann also nach dem ein solcher Flux gemessen wurden
die Daten der beiden Institute zusammenbringen und gemeinsam analysieren. Das IceCube hat dann genaue Daten zu den hochenergetischen Neutrinos 
und das DUNE genaue Daten zu den niederenergtischen. Beide Projekte sind bereits durch deren Organisationen mit viele Instituten 
verbunden, also würde sich eine projektübergreifende Kollaboration anbieten. 
Sehr bedauerlich ist es, dass beim DUNE-Experiment keine deutschen Institute mitmachen. Bei einem Experiment in einem solchen
Umfang sollten sich die deutschen Institute überlegen, ob sie nicht was dazubeitragen wollen.