Offener Brief an die Gruppe der Radioaktiven bei der Gauverein-Tagung zu Tübingen. \\
Physikalisches Institut \\
der Eidg. Technischen Hochschule\\
Zürch \\
Zürich, 4. Dez. 1930 \\
Gloriastrasse \\
Liebe radioaktive Damen und Herren, 
Wie der Ueberbringer dieser Zeilen, den ich huldvollst anzuhöhren bitte, 
Ihnen des näheren auseinandersetzen wird, bin ich angesichts der ``falschen'' 
Statstik der N- und Li-6 Kerne, sowie des kontinuierlchen beta-Spektrums auf 
einen verzweifelten Ausweg verfallen um den ``Wechselsatz'' (1) der Statistik 
und den Energiesatz zu retten. Nämlich die Möglichkeit, es könnten elektrisch 
neutrale Teilchen, die ich Neutronen nennen will, in den Kernen existieren, 
welche den Spin 1/2 haben und das Ausschliessungsprinzip befolgen und sich von 
Lichtquanten ausserdem noch dadurch unterscheiden, dass sie nicht mit Lichtgeschwindigkeit 
laufen. Die Masse der Neutronen müsste von derselben Größenordnung wie die Elektronenmassen
sein und jedenfalls nicht größer als 0,01 Protonenmasse. Das kontinuierliche beta-Spektrum 
wäre dann verständlich unter der Annahme, das beta-Zerfall mit dem Elektron jeweils noch 
ein Neutron emittiert würde derart, dass die Summe der Energien von Neutronen und Elekronen 
konstant ist. \\ \\
Nun handelt es sich weiter darum, welche Kräfte auf die Neutronen wirken. 
Das wahrscheinlichste Modell für das Neutron scheint mir aus wellenmechanischen Gründen 
(näheres weiss der Ueberbringer diese Zeilen) dieses zu sein, dass das ruhende Neutron ein 
magnetischer Dipol von einem gewissen Moment $\mu$ ist. Die Experimente verlangen wohl, 
dass die ionisierende Wirkung eines solchen Neutrons nicht grösser sein kann, 
als die eines gamma-Strahls und darf dann $\mu$ wohl nicht grösser sein als e*(10-13 cm). \\ \\
Ich traue mich verläufig aber nicht, etwas über diese Idee zu publizieren und wende mich 
erst vertrauensvoll an Euch, liebe Radioaktive, mit der Frage, wie es um den 
experimentellen Nachweis eines solchen Neutrons stände, wenn dieses ein ebensolches oder 
etwa 10mal größeres Durchdringungsvermögen besitzen würde, wie ein gamma-Strahl. \\ \\
Ich gebe zu, dass mein Ausweg vielleicht von vornherein wenig wahrscheinlich 
erscheinen wird, weil man die Neutronen, wenn sie existieren, wohl längst gesehen hätte. 
Aber nur wer wagt, gewinnt und der Ernst der Situation beim kontinuierlichen beta-Spektrum 
wir durch den Ausspruch meines verehrten Vorgängers im Amte, Herrn Debye, beleuchtet, der 
mir kürzlich in Brüssel gesagt hat: ``O, daran soll man am besten gar nicht denken, sowie 
an die neuen Steuern.'' Darum soll man jeden Weg zur Rettung ernstlich diskutieren. Also 
liebe Radioaktive, prüfet, und richtet. - Leider kann ich nicht persönlich in Tübingen 
erscheinen, da ich infolge eines in der Nacht vom 6. zum 7. Dez. in Zürich stattfindenden 
Balles hier unabkömmlich bin. - Mit vielen Grüssen an Euch, sowie auch an Herrn Back, 
Euer untertänigster Dienser \\ \\ 
gez. W. Pauli