\documentclass{artifacts/bll}
    %
    % Einstellungen der Schriftart
    %
    \usepackage[T1]{fontenc}
    
    \usepackage[light,math]{iwona}
    \usepackage{tabularx}
    \usepackage{amsmath}
    \usepackage{isotope} 
    \usepackage{xcolor}
    \usepackage{graphicx,wrapfig,lipsum}
    \usepackage{setspace} 
    
    % 
    % Blindtext zum Testen von Textausgaben
    %
    \usepackage{blindtext}    
    \usepackage{titlesec}  
    \usepackage{cite}
    \def\BibTeX{{\rm B\kern-.05em{\sc i\kern-.025em b}\kern-.08emT\kern-.1667em\lower.7ex\hbox{E}\kern-.125emX}}
    
    % 
    % Kopfzeile & Fußzeile
    %
    \pagestyle{fancy} 
    \renewcommand{\chaptermark}[1]{\markboth{#1}{}}
    \renewcommand{\sectionmark}[1]{\markright{#1}{}}
    \usepackage{etoolbox}
    
    %
    % Titelformat
    %
    \patchcmd{\chapter}{\thispagestyle{plain}}{\thispagestyle{fancy}}{}{}
    \titlespacing{\chapter}{0pt}{0pt}{0pt}
    \definecolor{gray75}{gray}{0.75}
    \newcommand{\hsp}{\hspace{20pt}}
    \titleformat{\chapter}[hang]{\Huge\bfseries}{\thechapter\hsp\textcolor{gray75}{|}\hsp}{0pt}{\Huge\bfseries}

    \usepackage[colorlinks, 
    pdfpagelabels,
    pdfstartview = FitH,
    bookmarksopen = true,
    bookmarksnumbered = true,
    linkcolor = black,
    plainpages = false,
    hypertexnames = false,
    citecolor = black] {hyperref}
    %%%%%%%%%%%%%%%%%%%%%%%%%%%%%%%%%%%%%%%%%%%%%%%%%%%%%%%%%
    \begin{document}
    
    \setlength{\parindent}{0em}

    %
    % Titelseite
    %
    \maketitle
    %
    % Vorstellung / Abstrakt
    %
    \begin{doublespace}
    \addcontentsline{toc}{chapter}{Kernthesen}
    \chead{Kernthesen}
    \input{chapters/Kernthesen}  
    \vfill  
    \pagebreak        
    \addcontentsline{toc}{chapter}{Arbeitsprozess}
    \chead{Arbeitsprozess}
    \input{chapters/Arbeitsprozess}
    \vfill     
    \end{doublespace} 
	\pagebreak
    
    %
    % Inhaltsverzeichnis     
    \addcontentsline{toc}{chapter}{Inhaltsverzeichnis}
    %
    \inhaltverzeichnis 
    
    \begin{doublespace}
    %
    % Inhalt
    % 
    \chapter{IceCube South Pole Observatory} 
    \vspace{8pt}
    \section{Was ist das IceCube-Projekt}
    \section{Warum ist es am Südpol ?}
    \section{Geschichte}
    \subsection{Konstruktion}
    \subsection{Erfolge}
    \section{Technik}
    \section{Funktionsweise}
    \section{Finanzierung}    
    \vfill  
    \pagebreak   
    \chapter{Neutrinoforschung}
\vspace{8pt}

\section{Das Neutrino}
Das Neutrino ist ein subatomares Teilchen der Leptonenklasse. Das Neutrino hat keine
elektrische Ladung und unterliegt somit nur der schwachen Wechselwirkung und der
Massenanziehungkraft. Nach dem Standardmodell ist das Neutrino ein punktförmiges Teilchen.
Es gibt 3 Generationen von Neutrinos mit jeweils anderer Masse.
Da Neutrinos ein Spin von $\frac{1}{2}$ haben, sind sie Fermionen. \\ \cite{Stoecker2000}
\begin{center}
    \begin{tabular}{ | l | c | } \hline
        Bezeichnung & Masse (MeV) \\ \hline
        Elektron-Neutrino & >7,3 $\cdot 10^{-6}$\\
        Muon-Neutrino & <0,27  \\
        Tau-Neutrino & <31 \\ \hline
    \end{tabular}
\end{center}
Die Masse wurde bis dato nicht genau bestimmt,
aber es wurden bisher Obergrenzen bestimmt.\\
Kosmische, solare, atmosphärische oder Geoneutrinos entstammen aus natürlichen Quellen bzw.
natürlichen Reaktionen.
Reaktorneutrinos und Beschleunigerneutrinos entstammen aus künstlichen Quellen bzw. künstlichen
Reaktionen. \\
Neutrinos können bei der Überprüfung der Plutoniumproduktion von Kernkraftwerken eingesetzt
werden, indem man die Antineutrinoemissionen misst. \cite{Krauter2006} \\
Insbesonders in der Astrophysik sind Neutrinos von hoher Bedeutung.
Da sie nur schwach wechselwirken durchdringen sie fast jede Materie und so kann man durch
Neutrinos Weltraumregionen untersuchen, die sonst aufgrund von Strahlungseinflüssen
nicht untersucht werden könnten.
Zudem ist die Masse von Neutrinos bedeutend für viele astrophysikalische Theorien. 

\subsection{Symmetrie}

\subsubsection{Händigkeit}

Es ist wichtig zwischen Chiralität und Helizität zu unterscheiden.
Die beiden Konzepte werden gerne vertauscht, da beide oft durch rechts und links ausgedrückt werden.
Die Helizität ist abhängig von der Drehrichtung eines Teilchens und des Spins eines Teilchen. \cite{Gold1958}
Die Helizität ist bei massereichen Teilchen vom Bezugssystem abhängig und bei masselosen Teilchen fest
definiert. Die Chiralität ist immer unabhängig vom Bezugssystem, ist also eine Lorentz-Invariante.
Bei masselosen Teilchen ist die Chiralität gleich der Helizität.\\
Nach bisherigen Versuchen haben Neutrinos eine negative Helizität (linkshändig) und seine
Antiteilchen eine positive Helizität (rechtshändig).
Auch wurde festgestellt, dass deren Helizität und Chiralität sich gleichen. Also könnte man davon ausgehen, dass
das Neutrinos masselos ist. \\
Dies steht natürlich im Widerspruch mit der Tatsache, dass Neutrinos eine Masse haben.
Dies ist auch im Moment noch ein offene Frage. Auch das IceCube-Projekt beschäftigt sich mit dieser Frage.
Eine mögliche Lösung wäre, dass das Neutrino wie andere Teilchen eine rechtshändige und linkshändige Version hat.
Ein rechtshändiger Neutrino würde nicht schwach wechselwirken, diese Neutrinos bezeichnet man als steriles
Neutrino. Man hofft, dass ein bestimmtes Mechanismus zwischen dem aktiven und sterilen Neutrino sowohl deren Masse
als auch die Symmetrien erklärt.

\subsubsection{Leptonenladung}

Das Neutrino hat ein positve Leptonenladung, während das Antineutrino eine negative Leptonenladung hat. Also kann man
zwischen Teilchen und Antiteilchen differenzieren, da sie verschiedene Quantenzahlen haben.\cite{Stoecker2000}
Diese Unterscheidung gibt es aktuell aber nur nach dem Standardmodell. Das zuvor erwähnte Mechanismus wird als See-Saw-Mechanismus
bezeichnet, dieses erfordet sogenannte Majorana-Spinoren, also im Grunde müsste die Leptonenladung von Neutrino und
Antineutrino gleich sein. Bis dato konnte man nicht experimentell bestimmen ob Neutrino und Antineutrino sich wirklich
unterscheiden.

\subsection{Geschwindigkeit und Masse}

Neutrinos haben nach momentanen Verständnis eine Masse. Nach dieser Annahme dürften Neutrinos nicht mit Lichtgeschwindigkeit
sich fortbewegen können. Nach der speziellen Relativitätstheorie gilt folgende Formel:
\begin{center}
    $E={\dfrac {m_0c^{2}}{\sqrt {1-{\frac {v^{2}}{c^{2}}}}}}$ \cite{Stoecker2000}
\end{center}
Würde ein Teilchen sowohl eine Ruhemasse $m_0 > 0$, als auch sich mit einer Geschwindigkeit $v=c$ fortbewegen,
dann würde sich eine Definitionslücke ergeben, da $x/0$ nicht definierbar ist.
\begin{center}
    $E={\dfrac {m_0c^{2}}{0}} $
\end{center}
Da die Annahme $m_0 > 0$ und $v=c$ nach der speziellen Relativitätstheorie keine Definition über die Energie des Teilchen
bei Bewegung zu lässt geht man davon aus die Annahme sei falsch.
Also bleiben nach der Theorie 2 mögliche Lösungen übrig. Entweder hat das Neutrino doch keine Ruhemasse, also es gibt
irgendein bisher nicht bekannten Masseeffekt oder das Neutrino bewegt sich nur subluminar.

\subsubsection{Erste Untersuchungen}

Eine erste Untersuchung der Geschwindigkeit von Neutrinos war ein Versuch von FermiLab in den 70ern.
Man konnte mit hoher Konfidenz feststellen, dass sich Neutrinos mit Lichtgeschwindigkeit fortbewegen.
Bei der Supernova SN1987A hat man diese Festellung nochmal bestätigt. Die Geschwindigkeit von $\overline{\nu}_e$
hat maximal ein Abweichung von $2x10^{-9}$ zur Lichtgeschwindigkeit. \cite{Longo1987}

\subsubsection{OPERA-Experiment}

Andere Messungen haben dies weiter bestätigt. Das OPERA-Experiment hat im Zeitraum von 2009-2011 Messungen bezüglich der Geschwindigkeit von Neutrinos
durchgeführt. Es wurde festgestellt, dass Neutrinos sich superluminal bewegen. Auf einem Weg von 743km kamen die Neutrinos
$60,7 \pm 14,3 ns$ früher an als wenn sie sich mit Lichtgeschwindigkeit fortbewegen würden. Mit einer $6\sigma$-Sicherheit
konnte man von einen sehr signifikaten Ergebnis reden. Diese Messungen wurden am 22.September 2011 in
einem Vordruckserver veröffentlicht. \cite{OPERA2011} \\
Im nächsten Jahr wurde dies aber korrigiert. Man hatte festgestellt, dass eine Kabelverbindung die gemessen Geschwindigkeiten
verändert hat. Es ergab sich nur eine Abweichung von $6.5_{-15.4}^{+15.7}\, ns$. In der weiteren Analyse kam man
zum Schluss, dass die Ergebnisse die Tatsache bestätigen, dass Neutrinos sich mit Lichtgeschwindigkeit bewegen.
Eine letzte Annahme könnte sein, dass zwar nach bisherigen Messungen Neutrinos sich mit großer Sicherheit mit
Lichtgeschwindigkeit fortbewegen, aber dies nur der Fall ist, weil deren Masse so gering ist, dass die Abweichung
zur Lichtgeschwindigkeit zu gering ist um sie messen zu können. Sie müsste kleiner als $2x10^{-9}$ sein.

\section{Geschichte}

\subsection{$\beta^{-}$ Zerfall}

Beim $\beta^{-}$-Zerfall gibt es folgende Reaktion:
\begin{center}
$n \rightarrow p + e^- + \overline{\nu}_e$
\end{center}
Vor der Entdeckung des Neutrino, hat man beim $\beta^{-}$-Zerfall nur die folgende Reaktion beobachtet:
\begin{center}
$n \rightarrow p + e^-$
\end{center}
Nimmt man diese Beobachtung kommt man auf den Energiehaltungssatz:
\begin{center}
$ E_e = E_n - E_p $
\end{center}
Man geht davon aus, dass das Neutron kein Impuls hat aufgrund der Laborbedingungen, also folgt
$p_n = 0$. \\
Aufgrund der Impulserhaltung ergibt sich folgendes:
\begin{center}
$ - p_e = p_{He} $
\end{center}
So kommt man zur Formel zur Berechnung der Energie des Elektrons:
\begin{center}
$ E_e = c^2 \frac{m_H^2 - m_{He}^2+m_e^2}{2m_H} $
\end{center}
$E_e$ beträgt somit $18,7 keV$. \\
Die gesamte Herleitung ist in der Quelle \cite{Horak2015} zu finden. \\ \\
Der entsprenchende Versuch lieferte jedoch nicht ein Linienspektrum für die Elektronenenergie,
man schlussfolgert also, dass die Elektronenenergie nicht konstant ist.
Man stellte jedoch fest, dass die höchste Elektronenenergie des Spektrum der Berechneten gleicht.
Aufgrund des Energieerhaltungssatz muss die restliche Energie irgendwie umgewandelt werden.
\cite{Horak2015} (siehe Anhang \ref{fig:b-zerfall}) \\ \\
Die erste mit dem Energieerhaltungssatz konforme Erklärung für dieses Phänomen kam mit einem Brief von
Pauli an die Teilnehmer der Gauverein-Tagung in Tübingen.
Im Brief postuliert er, dass dieses Energiespektrum aufgrund eines weiteren Teilchen
entsteht. \cite{Pauli1930} \\
Für Jahre konnte man keine Messung durchführen, welche dieses Teilchen beweisen würde.
Das postulierte Teilchen sollte aber elektrisch neutral sein und nur schwach wechselwirken.

\subsection{Reines-Cowan-Experiment}

Das postulierte Neutrino sollte in einem umgekehrten $\beta^{-}$-Zerfall mit einem Proton interagieren
und einen Elektron und Positron erzeugen.
\begin{center}
$\overline{\nu}_e + p \rightarrow e^+ + n$
\end{center}
Diese Reaktion hat ein Wirkungsquerschnitt $6 x 10^{-44} cm^2$, dieser ist 20 Magnituden kleiner ist als die entsprechende
Standardeinheit Barn ($10^{-24} cm^2$).
Durch den sehr kleinen Wirkungsquerschnitt ist die Reaktion sehr selten, also musste sie einfach identifizierbar
sein, da man nicht die wenigen Ereignisse verpassen wollte.
Die zwei Zerfallsprodukte sind Positronen und Neutronen. Die Positronen lässt man mit Elektronen interagieren
und so werden 2 Cadmiumen emittiert.
Diese 2 $\gamma$-Strahlen würden nicht reichen um die Reaktion leicht zu identifizieren.
Also lässt man noch das Neutron mit $Cd^{108}$ interagieren. Trifft ein Neutron auf $Cd^{108}$, regt es dieses an.
Diese angeregte Zustand ist instabil und das Cadmium gibt die Energie als $\gamma$-Strahlung frei. Dieser Prozess
passiert nicht sofort, also hat es eine Zeitsignatur.
Durch diese Kombination aus 2 $\gamma$-Strahlen und einem zeitverzögerten $\gamma$-Strahl kann man die Reaktion einfach
identifizieren und ist klar zu erkennen trotz Hintergrundstöreinflüssen. \\
Man benötigte dennoch ein hohen Neutrino Flux um die seltene Reaktion beobachten zu können. Dies konnte
man mit einem Atomreaktor erreichen, dieser sollte ein Neutrino-Flux von  $10^{12}-10^{13}$ Neutrinos
pro Sekunde pro Quadratzentimeter haben. \cite{Nave2017} (Anhang \ref{fig:c-experiment}) \\
Der entsprechende Versuch wurde in Hanford und schließlich in Savannah River durchgeführt, es wurden
durchschnittlich 3 Neutrinos pro Stunde gemessen. Die Ereignissanzahl war größer mit eingeschaltetem
Reaktor. \\
Die Ergebnisse aus dem Versuch legten nahe, dass Pauli mit seiner Neutrino-These korrekt lag.

\subsection{Homestake-Experiment}

In der Sonne gibt es Reaktionen, welche Neutrinos produzieren. 86\% dieser Neutrinos stammen aus der
Proton-Proton-Reaktion.
\begin{center}
$p + p \rightarrow d + e^- + \nu_e$
\end{center}
1968 konnten man den Neutrino-Flux der Sonne messen mit dem Homestake-Experiment.\cite{Cleveland1998}
Es wurde jedoch ein Defizit zwischen den berechneten Flux und den gemessen Flux festgestellt.
Es ist wichtig zu vermerken, das Homestake-Experiment konnte nur Elektron-Neutrinos messen .
Diese Problematik wurde bekannt als das Solarneutrino-Problem. Es wurde zum Beispiel vermutet, dass die Fusionprozesse
temporär ausgesetzt sind oder die Sonne ist kälter als man vermutet hatte. Das passte aber nicht zu den
helioseismologischen Daten, welchen mit dem Standard-Sonnenmodell übereinstimmten.

\subsection{Neutrino-Oszillation}

In den 70ern wurde vermutet, sollten Neutrinos eine Masse haben, dann würden sie zwischen den einzelnen Flavours
wechseln können. \cite{Gribov1969}
Bei den Proton-Proton-Reaktionen, welche in der Sonne stattfinden zerfallen Elektronen-Neutrinos.
Also könnten die fehlenden Neutrinos auf den Weg zur Erde den Flavour gewechselt haben und wären so nicht bei dem
Homestake-Experiment messbar.
2001 wurde das Problem dann endgültig gelöst. Das SNO konnte die Neutrino-Oszillation, also der Wechsel
zwischen einzelnen Neutrino-Flavours, nachweisen. Das Experiment bewies auch,
dass die Oszillation das Defizit von Elektron-Neutrinos erklärt.\cite{Ahmad2001}
Bei der Neutrinooszillation wechselt das Neutrino nach zurückgelegten Weg seinen Flavour.
Das Neutrino hat im Grunde 3 Massen-Eigenzustände $\nu_1,\nu_2,\nu_3$. Jedes der Flavours ist eine bestimmte
Überlagerung dieser 3 Massen-Eigenzuständen. Das Neutrino kann in entweder einem festen Flavour-Zustand sein, also
$\nu_e,\nu_\mu,\nu_\tau$ oder ein festen Massezustand. Mithilfe der Quantemechanik kann man die Eigenzustände als
Wellen betrachten. Also kann mithilfe der Schrödinger-Gleichung betrachten wie sich das Neutrino wandelt über ein
bestimmten Zeitraum.

\section{Forschung}

\subsection{Astrophysik}

Das GZK-Cutoff limiert wie stark energetisch komische Strahlung sein kann.
Der Cutoff wurde bei $6x10^{19} eV$ festgestellt und berechnet. \\
Dies limitiert die Beobachtung von komischer Strahlung bei Objekten die weiter als 100 Megaparsec sind.
Elektromagnetische Strahlung interagiert mit Staub- und Gaswolken, weshalb diese Signale abschirmen können.
Neutrinos jedoch haben kein GZK-Cutoff und interagieren nicht mit Stau- und Gaswolken, weshalb einige
komische Objekte nur mit Neutrinomessungen untersucht werden können. Supernovae setzen ein hohen Anteil
an Neutrinos und somit eignet sich die Beobachtung von Neutrinos für Supernovaeforschung. Das IceCube-Projekt
ist einer der wichtigsten Neutrinodetektoren für kosmische Neutrinos.

\subsection{Kosmologie}

In der Kosmologie ist kosmische Mikrowellenhintergrundstrahlung (3K-Strahlung) sehr wichtig gewesen um viele
Theorien, wie die Urknalltheorie, zu bestätigen. Die 3K-Strahlung entstand ungefähr 380 Tausend Jahre nach dem
Urknall. Es gibt einen entsprenchenden Counterpart mit Neutrinos. Der kosmiche Neutrinohintergrund ist wie die 3K-Strahlung
durch den Urknall entstanden.
Im frühen Beginn des Universums gab es viele Interaktionen zwischen Teilchen. Bei vielen Reaktionen entstehen Lichtquanten
und Neutrinos. Da Neutrinos kaum interagieren, können wir diese noch heute messen.
Die Neutrinos die dort enstanden interagierten nur mit anderen Teilchen in der ersten Sekunde in der das Universum
existierte. So ergibt sich eine Temperatur dieser von nur 1,95K. Umgerechnet liegen die Neutrinos im Energiebereich
von ungefähr $100-200 \mu eV$. Neutrinodetektoren können aber meisten nur ab den MeV-Bereich messen. Also
konnte man dies nicht direkt beweisen. Mit dem Planck-Weltraumteleskop konnte man aber die 3K-Strahlung so genau
messen und untersuchen, dass man über diese indirekt den kosmischen Neutrinohintergrund nachweisen konnte.
Die Messungen von Planck-Satellit war so genau, dass man darüber hinaus beweisen konnten, dass der Hintergrund eine
Temperatur von 1,96K hatte und das es nur 3 Neutrino-Flavours gibt ($e,\mu,\tau$). Die erwähnten Temperaturen
gelten nur unter der Bedingung, dass Neutrinos masselos sind. \cite{Siegel2016} Dies widerspricht aber der Tatsache,
dass nach aktuellen Messungen Neutrinos eine Masse haben. Also ist die Untersuchung des kosmichen Neutrinohintergrund
noch nicht vollendet.

\subsection{Zukünftige Forschung}

Das Neutrino hat durchaus die Teilchenphysik revolutionert und es dürfte auch in der Zukunft neue wichtige
Erkentnisse in der Teilchenphysik liefern. \\
ARIANNA und das Giant Radio Array for Neutrino Detection sollen auch wie das IceCube hochenergetische Neutrinoquellen
finden. Es gibt schon weitere Experimente in diesen Bereich, welche für ein Interesse der wissenschaftliche Gemeinde sprechen. \\
Ein kleiner Forschungzweig untersucht inwiefern man Neutrinos für Kommunikation durch solides Gestein nutzen könnte.
Es wurden auch schon entsprechende Versuche gemacht und man konnte erfolgreich Informationen mithilfe von Neutrinos durch
Gestein senden.

\subsubsection{DUNE}

Eines der großen Neutrinoexperimente die geplant werden ist das DUNE. Damit sollen Neutrinos produziert
und dann durch eine große Entfernung gesendet werden und dann an ein anderen Standort empfangen und gemessen werden.
(Anhang \ref{fig:dune}). \\ Das DUNE Experiment hat sich hohe Ziele gesetzt und will die Entstehung von Materie,
die Verbindung aller fundamentalen Kräfte und die Entstehung von schwarzen Löcher näher erklären.
Diese hohen Ziele haben unter anderem damit zutun, dass die Neutrinos möglicherweise das Standardmodell
durcheinanderbringen könnten. Im Moment hat das Standardmodell nur Dirac-Fermionen, also solche mit unterscheidbaren Antiteilchen.
Möglicherweise existieren jedoch Majorana-Fermionen, welche Antiteilchen haben mit gleichen Eigenschaften, also
kann man die nicht unterscheiden. Es gibt bereits Erweiterungen des Standardmodells mit Majorana-Fermionen, bisher
wurde keine dieser Erweiterungen experimentell bestätigt. \\ Das DUNE-Projekt könnte auch die
Materie-Antimaterie-Diskrepanz des frühen Universums erklären, welche zum heutigen Universum voller Materie
und mit kaum Antimaterie führte.

\section{Forschung am IceCube}

\subsection{Astrophysikalische Forschung}

Das größte Erfolg des IceCube ist bewiesen zu haben, dass mithilfe von Neutrinos man Bereiche des Universums
untersuchen kann, welche zuvor nicht untersuchbar ware, wie in 2.3.2 näher erklärt.
Dazu gehört, dass man 2017 durch die Kollaboration mit anderen Instituten zum ersten Mal
außergalaktische Neutrinos messen konnte. Diese stammten vom Blazar TXS 0506+056 aus der Orion-Konstellation.
Am Detektor wurde ein Alarm ausgelöst, weil man ein hochenergetisches (\textasciitilde 300 TeV) Muon-Neutrino maß. Anschließen
haben andere Institute mit deren Instrumenten nach der Quelle gesucht und diese als TXS 0506+056 identifiziert.
Dieses Ereignis wird insbesondere als Erfolg gewertet, weil damit zum ersten mal das Warnsystem erfolgreich erprobt
wurde. Mittlerweile wurde das Warnsystem erweitert, es gibt ein offenes System für Einzelereignisse und Mehrfachereignisse werden
durch Einzelvereinbarungen verschiedenen Instituten verteilt.
Über die Jahre wurden über 100 hochenergetische Neutrinos gemessen. Die gemessenen Energien ging von 100 TeV bis 10 PeV.
Die hochenergetischen Neutrinos scheinen zudem ein hohen Neutrino-Flux mit sich zu bringen.
\cite{WiMa2018}

\subsubsection{Gammastrahlenausbrüche}

Es wird vermutet, dass Gammastrahlenausbrüche die bisher gemesse hochenergetische kosmische Strahlung erklärt. Zudem wird auch
vermutet, dass Gammastrahlenausbrüche auch die hochenergetischen Neutrino-Events auslösen.
Das IceCube untersucht auch dies, da es in der Lage ist in den benötigten Energiebereich Neutrinos zu messen.
Bei diesen Gammastrahlenausbrüche kollidieren hochenergetische Protonen miteinander, also entstehen Pionen. \cite{Abbasi2011}
\begin{center}
$\,\pi ^{+}\to \mu ^{+}+\nu _{{\mu }} \;\;|\;\; \,\pi ^{-}\to \mu ^{-}+\overline {\nu }_{{\mu }}$ \\

$\pi ^{0}\to 2\gamma $
\end{center}
Geladene Pionen zerfallen unter anderem in Myon-Neutrinos und die neutralen Pionen in $\gamma$-Strahlen. Es ist naheliegend, dass
hochenergetische Neutrinos und hochenergetische $\gamma$-Strahlen aus den selben Quellen stammen \cite{Mueller2014}

\subsubsection{Supernovae}

Das IceCube-Projekt war zunächst nicht dafür gedacht Supernovae zu beobachten, da die Neutrinos, die bei einer Supernovae ausgestoßen werden
zu niederenergetisch sind damit das IceCube-Detektor sie messen kann. Doch die Photomultiplier müssten bei niederenergetischen Neutrinos
aus einem Supernovae-Ereignis kollektiv eine höhere Hintergrundrate aufweisen. Bei einer Supernova wird 99\% der Gravitationsenergie
als Neutrinos freigegeben, also entsteht ein hoher Neutrino-Flux.
Sollte ein Erhöhung der Hintergrundrate festgestellt werden wird das direkt an das Supernova Early
Warning System weitergegeben, damit andere Institute nach der Supernova suchen und möglicherweise mehr Daten sammlen. \cite{Eberhardt2017}

\subsubsection{Dunkle Materie}

Eines der größten Fragen der Astrophysik ist was ist Dunkle Materie. Auch hier gibt es entsprechende Versuche mit dem IceCube.
Eine möglicher Kandidat für dunkle Materie sind WIMP-Teilchen. WIMP's interagieren nur durch die schwache Wechselwirkung und der Gravitation.
Anders als Neutrinos sind sie jedoch sehr massereich. \cite{Gaensler2008}
Da WIMP's nur schwach wechselwirken sind sie auch schwer nachzuweisen.
Eine möglicher WIMP-Kandidat das Kaluza-Klein-Teilchen würde sich in Gestirnen ansammlen und wenn die Dichte hoch genug ist
würde sich selber annihilieren. Eines der Zerfallsprodukte wären Neutrinos im GeV/TeV Bereich.
Also würde eine erhöhte Anzahl an Neutrinos aus der Sonne nahelegen, dass das Kaluza-Klein-Teilchen
möglicherweise existiert. \cite{Abbasi2010}

\subsection{Glaziologie}

Das IceCube-Projekt fokusiert sich hauptsächlich auf Physik, durch seine besondere Lage im Eis des Südpol wurde es
auch für glaziologische Untersuchungen benutzt.
Das Eis, welches die optischen Sensoren des IceCube-Detektor umgibt, ist nicht 100\% durchsichtig, da es Impuritäten hat
wie zum Beispiel Staub. Die Daten von den optischen Sensoren müssen entsprechend korrigiert werden. Dieser Staub bietet
eine Einsicht in die Vergangheit des Erdklimas. Als die Sensoren ins Eis gebohrt wurden konnten man glaziologische
Untersuchungen machen.Man konnte vulkanische Asche finden. Man fand sogar Asche aus der Toba-Vulkaneruption, diese wird
in Anthropologischen Kreisen als mögliche Beeinflussung der menschlichen Ausbreitung diskutiert. Es wird in Zukunft auch
noch mehr Bohrungen geben, da die Daten aus dem IceCube ein entsprenchendes Interess weckten. \cite{IceCube2013}

\section{Interpretation von Neutrino-Ereignissen}

\subsection{Spursignatur\,/\,$\mu$-Signatur}
\begin{center}
    \includegraphics[scale=0.43]{images/track.png} \\
    \textit{[Fig. 1]}
\end{center}
Das $\nu_\mu \,/\, \overline{\nu}_\mu$ hinterlässt Myonen. Wenn die Myonen durch den Detektor passieren
hinterlassen sie Cherenkov-Licht, da sie in Relation zur Lichtgeschwindigkeit im Eis sich schneller fortbewegen.
Da wie in \textit{[Fig 1]} ersichtlich wird das eine Spursignatur hinterlässt ist es einfach zu bestimmen aus
welcher Richtung das Myon kommt. Es wird eine Lineare Regression angefertigt. Das von den optischen Sensoren aufgenomme
Licht steht in Korrelation zur Myonenenergie.\cite{Halzen2012} \cite{Martinez2016}
\newpage

\subsection{Kaskadensignatur\,/\,$e$-Signatur}
\begin{center}
    \includegraphics[scale=0.45]{images/cascade.png} \\
    \textit{[Fig. 2]}
\end{center}
Das $\nu_e \,/\, \overline{\nu}_e$ hinterlässt eine Kaskade. Wenn es im Detektor wechselwirkt hinterlässt es einen Teilchenschauer.
Die Teilchen hinterlassen Cherenkov-Licht,  da sie in Relation zur Lichtgeschwindigkeit im Eis sich schneller fortbewegen.
Wie man in \textit{[Fig. 2]} entsteht dadurch eine kugelförmige Signatur. Das von den optischen Sensoren aufgenomme
Licht steht in Korrelation zum der freigegeben Energie durch die Wechselwirkung mit dem Detektor. \cite{Halzen2012}
\newpage

\subsection{Double-Bang-Signatur\,/\,$\tau$-Signatur}
\begin{center}
    \includegraphics[scale=0.45]{images/doublebang.png} \\
    \textit{[Fig. 3]}
\end{center}
Bei der Double-Bang-Signatur hat ein $\nu_\tau \,/\, \overline{\nu}_\tau$ mit einer Energie über 1 PeV eine erste Kaskade verursacht.
Bei dieser Kaskade entsteht ein $\tau$, welches dann zerfällt und eine zweite Kaskade verursacht.
Wie man in \textit{[Fig. 3]} entstehen dann 2 Kaskaden. Einmal gibt es eine rote Kaskade und dann eine grüne Kaskade.
Die rote Kaskade ist die erste Kaskade und ist dann mit der grünen zweite Kaskade verbunden. Diese Signatur ähnelt
sehr der Spursignatur, man kann sie aber an der ungleichmäßigen Größe der Signale unterscheiden. \cite{Halzen2012} 
    \chapter{Inwiefern hat das IceCube-Projekt seine Ziele erreicht} 
    \vspace{8pt}
    \section{Neutrinos aus komischer Strahlung}
    \section{Neue Erkenntnisse über Neutrinos}    
    \vfill  
    \pagebreak     
    \chapter{Rolle des IceCube-Projekts in der Wissenschaftsdiplomatie} 
    \vspace{8pt}

    \section{Finanzierung und Kooperationen}
    
    Die Kosten des Projekts beliefen sich auf etwa 279 Millionen US-Dollar. 
    Die Kosten für das Projekt wurden hauptsächlich von der \grqq National Science Foundation\grqq{} getragen. 
    Das sind etwa 242 Millionen US-Dollar, die von der \grqq National Science Foundation\grqq{}  übernommen wurden. 
    Die weitere Finanzierung erfolgte durch Universitäten und Institute aus vielen verschiedenen Ländern 
    und einigen Forschern, neben den USA waren auch Länder wie Deutschland, die Niederlande und Belgien 
    bei der Finanzierung vertreten. Das Bundesministerium für \grqq Bildung und Forschung\grqq{} und die 
    \grqq Deutsche Forschungsgemeinschaft\grqq{} (DFG) subventionierten die Konstruktion des Observatoriums. 
    Aber auch beispielsweise das \grqq Institut zur Förderung von Innovation durch Wissenschaft und 
    Technologie in Flandern\grqq{}, welches sich in Belgien befindet, leistete finanzielle Unterstützung.
    Deutschland nimmt eine wichtige Rolle beim IceCube-Projekt ein. Des Weiteren ist DESY für die Prüfung 
    von 1250 optischen Modulen tätig gewesen oder auch für die Datenanalyse lieferten sie wichtige Beiträge. 
    Beispielsweise waren sie dafür zuständig die Software zu schreiben und elektrische Komponenten zu 
    entwickeln. Nicht zu vergessen ist, dass DESY stets an neuen Methoden für den Nachweis von Teilchen 
    arbeitet. Die \grqq National Science Foundation\grqq{} hielt es für angebracht der Universität von 
    Wisconsin-Madison die Führung zu übertragen, da diese Universität bereits mit dem Projekt AMANDA 
    beauftragt gewesen ist und die Ergebnisse zufriedenstellend waren. Sie hatten somit Erfahrung 
    im Bereich der Forschung von (kosmischer) Höhenstrahlung \cite{FAQ13}.

    \section{IceCube-Kollaboration}

    Etwa 300 Physiker bilden die IceCube-Kollaboration, darunter befinden sich Physiker, 
    Informatiker, Ingenieure etc. Seit November 2017 besteht sie aus 50 Einrichtungen in 12 Ländern, 
    welche an dem Detektor arbeiten, in dem sie ihn beispielsweise auf Funktionalität der Sensoren 
    überprüfen und die Messdaten auswerten, aber viele Mitglieder waren auch der Konstruktion des Detektors 
    beteiligt. Die Aufgabenbereiche sind unzählig, einige werden direkt an der Südpolststion ausgeführt 
    und andere forschen von ihren Universitäten aus, da IceCube nicht die einzige Aufgabe der 
    IceCube-Kollaboration darstellt. Ein Beispiel für ein deutsches Forschungsinstitut ist das 
    \grqq Deutsche Elektronensynchrotron\grqq{} (DESY), aber auch viele deutsche Universitäten wie die Technische 
    Universität München, die Ruhr-Universität Bochum oder auch die Johannes Gutenberg Universität sind an 
    diesem Projekt beteiligt und ebenso ist auch die Humboldt-Universität zu Berlin eine maßgeblich beteiligte 
    Universität \cite{DeFor13} \cite{FAQ13}.

    \section{Wissenschaftliche Bedeutung}

    Dieses Projekt ermöglicht eine Zusammenarbeit von vielen Universitäten und Instituten, 
    die eng miteinander agieren, um eine intensive und effiziente Forschung zu gewährleisten. 
    Es ist sinnvoll, dass möglichst viele Wissenschaftler zusammenarbeiten, da sich so viele Menschen mit 
    unterschiedlichen Arbeitsmethoden und Denkweisen aufeinandertreffen. Es ist möglich, dass diese voneinander 
    lernen können und zusammen den idealsten Weg finden, um Probleme zu lösen und die Forschung zu verbessern. 
    Außerdem stellt eine wissenschaftliche Zusammenarbeit auch eine Brücke zwischen zwei Gesellschaften dar, 
    um gemeinsame Strategien zur Überwindung globaler Differenzen und Problematiken zu konstruieren. 
    Nicht unerwähnt sollte die Tatsache bleiben, dass einige Forschungsstationen auch zusammenarbeiten und 
    nicht ausschließlich Konkurrenzdenken betreiben. Ein Beispiel dafür ist die Kooperation von IceCube mit 
    HAWC (ebenfalls ein auf Tscherenkow-Strahlung basierendes Neutrinoteleskop). Die verschiedenen 
    Kollaborationen haben Einfallsrichtungen der kosmischen Strahlung bei gleicher Energie in 
    unterschiedlichen Himmelrichtungen abgeglichen. Das Ziel bestand darin die Ausbreitung der 
    kosmischen Strahlung, mit einer geringeren Energie von etwa 10TeV, zu untersuchen, um die 
    Anisotropie besser zu verstehen. Das findet natürlich nicht ausschließlich bei IceCube statt, 
    aber IceCube ist ein passendes Beispiel dafür, dass internationale Zusammenarbeit, bei IceCube 
    sind Wissenschaftler aus zwölf Ländern vertreten, einige Erfolge bringt. Dies lässt sich vor allem 
    durch die wissenschaftlichen Leistungen festmachen, die durch IceCube geleistet wurden \cite{DeFor13} \cite{IceHa18}.

    
    \vfill  
    \pagebreak      
    \chapter{Vergleich zu anderen Forschungsstätten} 
\vspace{8pt}

\section{Super-Kamiokande}

Der Super-Kamiokande steht in der japanischen Stadt Hida, es handelt sich bei diesen um einen
Cherenkov-Detektor. Er ist der Nachfolger des Kamiokande, welches in seiner letzen Ausführung,
des Kamiokande-III, 1995 endgültig außer Betrieb ging.
Einer der Hauptmotivationen des Kamiokande-Projekts war es den Protonzerfall näher zu untersuchen. 
Da das originale Kamiokande in diesen Bereich keine Erfolge lieferte entschied man sich für 
den Bau des Super-Kamiokande. \cite{Fukanda2003}

\subsection{Geschichte}

Die Datenaufnahme wurde beim Super-Kamiokande im April 1996 begonnen. Es durchging immer wieder 
Reparaturen und wurde schon mehrmals nachgerüstet. \cite{Fukanda2003} Im November 2001 
geschah ein Unfall, ungefähr 7000 PMTs (optische Sensoren)implodierten. Es wurde vermutet, 
dass die Implosion einer PMT eine Schockwelle im Wasser verursachte, welche die andere PMT's
implodieren ließ. Der Super-Kamiokande musste schnell wiederaufgebaut werden, weshalb man sich 
entschied nicht alle PMTs zu reparieren, sondern man beließ es bei der Hälfte, so konnte der 
Betrieb schnell wieder aufgenommen werden. \cite{Cartlidge2001}

\subsubsection{Neutrinooszillationen}

Neutrinooszillationen sind ein interessantes theoretisches Konzept, experimentell wurden sie 
aber 1998 zum ersten Mal vom Super-Kamiokande nachgewiesen. 
Bei der Wechselwirkung zwischen kosmischer Strahlung und Atomkernen in der Atmosphäre 
entstehen atmosphärische Neutrinos als Zerfallprodukt. Dort dominieren zwei Reaktionen:
\begin {center}
$\tau \rightarrow \mu + \nu_\mu$\\
$\mu \rightarrow e + \overline{\nu}_\mu + \nu_e$
\end {center}
Also müsste sich logischerweise ein Verhältnis von 2$\nu_mu/\overline{\nu}\mu$:1$\nu_e/\overline{\nu}_e$ 
ergeben. Beim Super-Kamiokande wurde diese Annahme überprüft, man kann aber auf ein Verhältnis 
von 1:1. Dieser Unterschied konnte nicht durch statische Unsicherheiten oder experimentelle 
Unsicherheiten erklärt werden. Wenn man aber eine Neutrinooszillation von 
$\nu_\mu \rightarrow \nu_tau/\nu_?$ mitbedenkt, dann würde sich dieses Verhältnis erklären lassen. 
Also könnte dieser Effekt durch Neutrinooszillationen erklärt werden. \cite{FUKUDA2003} MacDonald und Takaaki 
Kajita haben 2015 ein Nobelpreis bekommen für den deren Arbeit an Neutrinooszillation, wobei
Kajita hauptsächlich am Super-Kamiokande gerabeitet hat und MacDonald an der SNO-Kollaboration.

\subsubsection{Weitere Experimente}

Um diese Entdeckung zu untermauern wurde anschließend das K2K-Experiment durchgeführt. Bei diesem Experiment
wurden $\nu_\mu$ vom KEK über eine 250km lange Strecke gesendet, das Endziel war der Super-Kamiokande. 
Das Super-Kamiokande konnte weiterhin kein $nu_\tau/\overline{\nu}_\tau$ registrieren. \cite{Collaboration2002}
Wenn die $nu_\mu/\overline{\nu}_\mu$ nicht oszillieren dann würde man $158,1^{+9,2}_{-8,6}$ gemessene Ereignisse 
am Super-Kamiokande erwarten. Tatsächlich wurden aber 112 gemessen. Die Wahrscheinlichkeit, dass die 
ohne Neutrinooszillation passiert ist liegt bei 0,0015\%. Man kann also mit hoher Sicherheit annehmen, 
dass die Resultate durch Neutrinooszillationen erklärt werden. \cite{Ahn2006}\\
Mit dem Super-Kamiokande wurde mit K2K und der Untersuchung von atmosphärischen Neutrinos nur die Oszillation
zwischen $\nu_\mu - \nu_\tau$. Wenn auch seltener gibt es auch eine Oszillation zwischen $\nu_\mu$ und 
$\nu_e$. Das T2K sollte das untersuchen. Beim J-PARC wurden wie beim vorherigen Experiment $v_\mu$ über 
eine 295km lange Strecke gesendet. Die Intesität wurde beim T2K-Experiment erhöht um 2 Magnituden.\cite{Yuichi2006}
Aktuell läuft das T2K-Experiment immer noch, also gibt es noch kein entsprechendes Abschlussbericht, doch
es wurden zwei Haupterfolge gemeldet. \\
Zum einen war der eine Erfolg der Beweis, dass $\nu_\mu$ sich tatsächlich in $\nu_e$ wandlen können. Mit diesem Erfolg
hat sich auch der Fokus des T2K-Experiment auf eine Untersuchung von CP-Verletzungen. \cite{T2K2017}
Es wurde auch eine mögliche CP-Verletzung gefunden, da man ein Unterschied zwischen der Oszillation von Neutrinos
und Antineutrinos entdeckt. Die Ereignisanzahl ist zu gering, weshalb man noch bis 2026 weiter Daten sammlen muss
um dies mit einer höheren Konfidenz zu bestätigen. \cite{T2K2013} 

\subsection{Hyper-Kamiokande}

Die PINGU-Erweiterung beim IceCube-Projekt scheiterte, währenddessen wurde die Erweiterung für das Kamiokande genehmigt. 
Diese Erweiterung wird ein deutlich größeren Cherenkov-Detektor benutzen, auch die PMTs wurden verbessert. Das Tankvolumen
steigert sich um das 20fache im Vergleich zum Super-Kamiokande. Es sollen auch 2 Tänke gebaut werden, wobei der erste
Tank in Betrieb gehen soll bevor der zweite Tank fertig gebaut wird.
Auch hier sollen $\nu$ mithilfe des J-PARC über eine lange Strecke gesendet. Also handelt es sich um die nächste
Generation des T2K-Experiments. Mit dem neuen Detektor wird man auch wieder atmosphärische Neutrinos und astrophysikalische
Neutrinos untersuchen. 
Mit dem neuen Detektor bietet sich eine erneute Untersuchung, da durch die
erhöhte Sensivität mehr Ereignisse messen kann und man kann Parameter genauer bestimmen.
Der zweite Tank des Hyper-Kamiokande soll nach Plan in Südkorea gebaut werden. Es hat sich herausgestellt, dass
der von J-Parc ausgesendete Neutrinostrahl in Südkorea auftrifft und somit wäre es möglich ein
Detektor dort zu bauen, welches Neutrinos, welche eine noch größere Strecke hinter sich haben,
messen würde. Damit kann man Neutrinooszillationen noch besser untersuchen.  
Es dürfte ungefähr 10 Jahre brauchen bis der erste Tank in Hida fertig gebaut wird, nachdem dieser läuft 
soll der Bau des zweiten Tanks in Sükorea nur noch 6 Jahre brauchen. Ob es tatsächlich in Korea gebaut werden
soll wird momemtan untersucht. \cite{Lodovico2017}
Der Bau des Hyper-Kamiokande soll 2020 beginnen. \cite{HyperK2018}

\subsection{Vergleich}

\subsubsection{Konstruktion}

Sowohl der IceCube-Detektor als auch der Kamiokande benutzen Cherenkov-Strahlung um Neutrinos zu messen.
Beim Kamiokande ist der Bau deutlich aufwendiger wie man am Hyper-Kamiokande sehen kann, man muss geologische Untersuchungen 
machen, man muss eine entsprechende Stelle aushöhlen und einen Tunnel zu der Stelle graben, dann noch ein Tank bauen, 
Wasser purifizieren und es in den Tank füllen. \cite{Lodovico2017} 
Beim IceCube-Detektor musste man stattdessen tiefe Löcher ins Eis schmelzen. Man braucht 2 Tage um das Loch zu schmelzen
und 11 Stunden um den optischen Sensor anzubringen. Problematisch ist der Transport von Materialien von den entsprechenden
Produktionsstätten zum Südpol.  \\
2001 begann die Planung für den IceCube-Detektor. Innerhalb von 2 Monaten nach Baubeginn im Jahre 2004, war der IceCube-Detektor 
betriebsbereits, aber es hat 6 Jahre gebraucht, damit es tatsächlich in seiner vollen Größe fertig war. 
Das Super-Kamiokande begann mit der Planung 1991 und wurde 1996 vollendet.
Zwar war die gesamte Bauperiode beim IceCube gering, da man es aber über die Jahre verteilte, also man hat diesen in Phasen
gebaut, dauerte es länger als beim Super-Kamiokande.  \\
Auch waren die Baukosten entsprechend höher. Beim Super-Kamiokande wurden 100M\$ vorgesehen und der IceCube hat tatsächlich 
279M\$ gekostet. 
Die Konstruktion des IceCube-Detektors war zwar einfacher, aber auch teurer. 

\subsubsection{Wissenschaftlicher Fokus}

Die größten Erfolge des Super-Kamiokande liegen im Bereich der Neutrinooszillation, beim IceCube war es hingegen die 
Etabilierung eines Supernovae-Frühwarnsystem und die erfolgreiche Erprobung am Blazar TXS 0506+056,die bisher größte astrophysikalische 
Neutrino-Quelle die gefunden wurde, einer deren großen Erfolge. \\
Man erkennt ein klaren Unterschied an den thematischen Fokus der beiden Detektoren. Während das Super-Kamiokande sich der 
Untersuchung der Eigenschaften von Neutrinos widmet, ist der IceCube-Detektor mehr auf die astrophysikalische Herkunft von
Neutrinos fokussiert und die Untersuchung der Eigenschaften von Neutrinos im astrophysikalischen Zusammenhang. 
Durch die besondere Lage macht das IceCube-Projekt auch glaziologische Untersuchungen, das Super-Kamiokande hat keiner
interdisziplinäre Untersuchungen. \\
Das Super-Kamiokande hat deutliche Fortschritte in der Teilchenphysik geleistet, das IceCube konnte in der Astrophysik
aber bereits beweisen, dass es in seinem Fachbereich was leisten kann. Auch die wichtigen Leistungen des Super-Kamiokande kamen 
nicht über Nacht sondern aber auch viele Jahre Forschung gebraucht. Das IceCube musste bevor es anfangen kann wichtige Beiträge
zur Astrophysik zu leisten noch beweisen, dass sein eigentliches Konzept auch viabel ist. \\
Es hat dennoch bereits viele Daten über Neutrinos gesammelt und in Kooperation mit anderen Instituten auch kleinere Fortschritte
in der Neutrinophysik geleistet. 

\section{DUNE}

Das DUNE ist ein in Planung befindendes Neutrinoexperiment. Es besteht aus 2 Neutrinodetektoren und einen Protonenbeschleuniger 
für die Produktion von Neutrinos. Hinter dem DUNE-Experiment steht eine große Kollaboration, die drei wichtigsten Institute
sind das CERN in der Schweiz, das Fermilab in Illnois und das Sanford Lab in South Dakota.  

\subsubsection{Aufbau}

Das DUNE-Experiment ist in Near-Site-Facilities(NSFs) und Far-Site-Facilities(FSFs) unterteilt, wobei die NSFs in der Nähe des 
Protonenbeschleunigers sind und die FSFs 1300 Kilometer vom Protonenbeschleuniger entfernt.
Zu den NSFs gehören der Fermilab mit seine Protonenbeschleuniger und der anschließenden Neutrinoproduktion, auch
gibt es eine Neutrinodetektor direkt am Fermilab. Zu den FSFs gehört ein weiterer Detektor in der Sanford Underground
Research Facility (Siehe \ref{fig:dune})

\subsubsection{Vorgeschichte}

Zunächst sollte das DUNE-Experiment in us-amerikanischer Einzelarbeit gebaut werden. Es hieß damals auch noch LBNE. \cite{Moore2015}
LBNE wurde dan aufgelöst und durch das DUNE-Experiment abgelöst, man wollte mit anderen Instituten zusammenarbeiten. 
Das LBNE sollte näher an der Oberfläche sein als das DUNE-Experiment. 2015 wurde das LBNE durch das ELBNF abgelöst und später auch
in das Dune-Experiment umbenannt. Der Bau begann 2017, 2018 konnten schon erste Messungen geführt werden. 2026 soll der Bau abgeschlossen
werden. \cite{Fermilab2015}

\subsection{Untersuchungsthemen}

Dieses Experiment soll in vielen Bereichen der Physik agieren. Man will astrophysikalische Untersuchungen machen zu Supernovae,
Untersuchungen in der Teilchenphysik zu der Grand Unified Theory (Vereinigung aller Grundkräfte inkl. Gravitation) und 
zu CP-Verletzungen. Auch soll die Masse des Neutrinos weiter untersucht werden. \cite{Brailsford2018}

\subsubsection{Supernovae}

Das DUNE-Experiment soll im Vergleich zu anderen Neutrino-Detektoren viel besser niederenergtische Teilchen messen. Damit
kann man die in einem Event freigesetzte Energie deutlich genauer bestimmen. Bei Supernovae werden zu deren Beginn eher $\nu_e$
emittiert und in der späteren Phase $\overline{\nu}_e$. Das DUNE-Experiment kann besser $\nu_e$ messen als $\overline{\nu}_e$ und
wird sich deshalb dafür eignen eher die frühe Phase von Supernovae zu untersuchen. Insbesondere erhofft man sich, dass mithilfe
von Daten über die Neutrinofluxe von anderen Detektoren, wie den Hyper-Kamiokande, man die Emission von Neutrinos bei der Supernovae
besser versteht. \cite{Ankowsi2016}

\subsubsection{Grand Unified Theory}

Nach einigen GUTs sollten Protonen zerfallen. Wie zum Beispiel nach dieser Reaktion:
\begin{center}
$ p \rightarrow K^+ + \overline{\nu} $
\end{center}
Einer der Zerfallsprodukte wäre ein Antineutrino. Das kann vom DUNE-Experiment gemessen werden.
Das K-Meson würde nach 12,8ns weitere zerfallen: 
\begin{center}
    $ K^+ \rightarrow \mu^+ + \nu_\mu $ /  $ K^+ \rightarrow \tau^+ + \tau_0 $
\end{center}

Nach theoretischen Überlegungen lässt sich auch bestimmen wie das entsprechende Spektrum aussieht. 
So müsste man zwei Peaks bei 257 MeV und 459 MeV sehen. Beim LENA-Projekt könnte man dann im
Laufe von 10 Jahren ungefähr 10 solcher Ereignisse sehen, sollten Protonen zerfallen, beim
DUNE gibt es noch keine genaue Angaben wie sensibel ist bei Protonenzerfällen. Da sie, die Messung
eines solchen deren Ziel ist, sollten das DUNE-Experiment entsprechend sensible Detektoren bauen. \cite{Undagoitia2008}

\subsubsection{CP-Verletzungen}

Wie beim Super-Kamiokande will man beim DUNE-Experiment CP-Verletzungen anhand von 
Neutrinooszillation nachweisen. 
Auch hier soll die Antineutrinooszillation mit der Neutrinooszillation verglichen werden. 
Das DUNE hat aber den Vorteil, dass es eine wesentlich längere Streck zwischen den Detektor und der Neutrinoproduktion
hat im Vergleich zum Super-Kamionkande. Nach 14 Jahre Laufzeit werden die vom DUNE gemessenen oder nicht gemessenen
Unterschied mit einer $3-\sigma$-Signifikanz stimmen. Sollte keine Effekt gemessen werden, dann wird das
DUNE-Experiment aussagen können wie groß die CP-Verletzung noch sein darf. 
\cite{Acciarri2016}

\subsubsection{Weitere Themen}

Das sind die vom Projekt selbst vorgenommenen Ziele, doch der große Umfang des Projekts und der Kollaboration führen dazu, dass
es noch weitere Überlegungen gibt das DUNE-Experiment in noch mehr Bereichen zu nutzen. 
Eine dieser Überlegungen ist mit dem DUNE nach sterilen Neutrinos zu suchen. Diese, wie zuvor erklärt, interagieren nur
durch Gravitation, das mach deren Messung besonders schwer. Beim DUNE sollen auch die Parameter der Neutrinooszillation
genauer gemessen werden, sollte diese nicht den theoretisch progonsizierten Werten, deutet das daraufhin, dass
das theoretische Verständnis des Neutrinos möglicherweise überarbeitet werden muss, dazu könnte eine Erweiterung
des Standardmodell um sterile Neutrinos gehören.\cite{Berryman2015}
Eine Erweiterung der Neutrinophysik könnten auch nicht standardmäßige Neutrinodetektoren bieten. Die Sensivität des DUNE sollte
ausreichen um zu bestimmen, ob es diese nicht standardmäßige Neutrinointeraktionen überhaupt gibt, es könnte
sogar reichen um die neuen Parameter zu bestimmen. Wenn die Datenlage gut ist, könnte man auch die Interaktion
näher bestimmen und gegebenfalls von sterilen Neutrinos unterschieden werden. Es wird erwartet, dass man
dennoch mehr Daten brauchen wird, als das DUNE liefern wird. \cite{Gouvea2016}

\subsection{Vergleich}

\subsubsection{Konstruktion}

Wie auch schon beim Super-Kamiokande ist der Bau deutlich aufwendiger. Zudem wird das DUNE-Experiment deutlich teurer
werden, als das IceCube-Projekt. Es wird nach jetztigen Schätzunden alleine der us-amerikanischen DOE 1,3-1,9 Billionen
Dollar kosten. 
Die Planungsphase für das DUNE-Experiment begann 2015 und der Bau soll 2026 fertig sein, das liegt ungefähr im gleichen
Rahmen wie beim IceCube, wobei das IceCube innerhalb von 3 Jahren in Betrieb ging, doch das DUNE-Experiment kann seinen Betrieb
erst nach Bauende beginnen. Hier zeigte sich auch wieder, dass die phasenweise Bauart des IceCubes den Vorteil hat, dass durch
diese es möglich ist relativ schnell mit Messungen zu beginnen.

\subsubsection{Wissenschaftlicher Fokus}

Das DUNE-Experiment hat sich ganz offensichtlich höhere Ziele gesetzt als das IceCube-Projekt. Anders als beim
Super-Kamionkande ist die Überscheidungen zwischen den beiden Detektoren größer. Sowohl das IceCube-Projekt als
auch das DUNE betreiben astrophysikalische Forschung im Bereich der Supernovae. Auch untersuchen beide die Masse
von Neutrinos, wobei fast alle Neutrino-Detektoren das untersuchen, dies eine sehr grundlegende Fragestellung ist, 
welche immernoch nicht genau beantwortet wurde. \\
Bei der Untersuchung von Supernovae könnten sich die beiden eine Rolle spielen, das IceCube hat bewiesen, dass es die
Neutrinofluxe aus Supernovae wahrnehmen kann und rechtzeitig andere Institute informiert über den Fund. So könnte
das IceCube ein Neutrinoflux einer Supernovae wahrnehmen und das DUNE-Experiment kann dann versuchen die niederenergtischen
Neutrinos zu messen, wobei es sein kann, dass keine mehr gemessen werden können, da wie zuvor festgestellt werden die niederenergtischen
Neutrinos zu Beginn der Supernova ausgestrahlt. \\
So eine zusammenarbeit könnte, aber auch ohne Warnsystem funktionieren. Beide Detektoren sollten den Neutrinoflux einer
Supernova wahrnehmen, beim IceCube wurde das bereits gemacht, auch spricht beim Aufbau der DUNE-Detektoren nichts dagegen, dass
sie wie das Super-Kamiokande Neutrinofluxe aus Supernovae wahrnehmen können. Man kann also nach dem ein solcher Flux gemessen wurden
die Daten der beiden Institute zusammenbringen und gemeinsam analysieren. Das IceCube hat dann genaue Daten zu den hochenergetischen Neutrinos 
und das DUNE genaue Daten zu den niederenergtischen. Beide Projekte sind bereits durch deren Organisationen mit viele Instituten 
verbunden, also würde sich eine projektübergreifende Kollaboration anbieten. 
Sehr bedauerlich ist es, dass beim DUNE-Experiment keine deutschen Institute mitmachen. Bei einem Experiment in einem solchen
Umfang sollten sich die deutschen Institute überlegen, ob sie nicht was dazubeitragen wollen. 
        \section{Antares (Astronomy with a Neutrino Telescope and Abyss environmental Research)}

    Bei Antares handelt es sich ebenfalls um ein Neutrinoteleskop, welches Neutrinos kosmischer Herkunft 
    untersucht. Die Ziele, die beide Teleskope verfolgen sind nahezu identisch, wie die Ermittlung der 
    Strahlungsquellen kosmischer Höhenstrahlung. Ihre Funktionsweisen sind sich ebenfalls sehr identisch. 
    Es wird sich bei Antares zu Nutze gemacht, dass in seltenen Fällen die Neutrinos mit transparentem 
    Material wechselwirken. Antares befindet sich im Mittelmeer und man setzt darauf, dass die Neutrinos 
    mit dem Wasser wechselwirken. Ähnlich wie bei IceCube entstehen daraus dann Sekundärteilchen wie Myonen, 
    die dann mithilfe von Detektoren registriert werden. Beide Detektoren machen sich die seltene 
    Wechselwirkung von H2O mit Neutrinos zu nutze. Auch bei Antares entsteht Tscherenkowlicht. 
    Wie beim IceCube wurden auch hier einige Stränge mit optischen Modulen im Medium platziert. 
    Im Gegensatz zu IceCube wird es hier ermöglicht die Detektoren wieder zu erreichen und zu warten für den 
    Fall, dass sie ausfallen sollten. Antares stellt auch eher einen Prototyp für einen im Aufbau befindlichen 
    Teilchendetektor dar, nämlich KM3NeT. Dieser Detektor wird Antares ablösen, da Antares nur Messungen im 
    Bereich von 10 GeV bis 100 GeV wahrnehmen kann. Für höherenergetische Messungen ist das Teleskop zu 
    ungenau, was eigentlich essentiell ist, da höherenergetische Neutrinos eher Aufschluss darüber geben 
    woher Neutrinos stammen und welche Bedeutung sie haben. IceCube im Vergleich misst vor allem Daten im 
    Bereich von $\geq$200 GeV, was einige irrelevante Messungen wie irdische Neutrinos vernachlässigt. KM3Net 
    soll dann wesentlich größer werden mit 600 Strängen und 12000 Photomultiplern und ein größeres Spektrum 
    an möglich zu messbaren Energien. Beide Detektoren richten den Fokus auf Myonneutrinos, da sie nach 
    der Kernreaktion mit dem Medium ihre Bewegungsrichtung beibehalten. Sowohl IceCube als auch Antares 
    werden für andere Forschungen außerhalb der Neutrinoforschung eingesetzt. Durch die Positionierung 
    im Mittelmeer lässt es auch biologische Untersuchungsmöglichkeiten zu, wie die Erforschung der 
    Tierwelt oder von Biolumineszenz. IceCube bittet keine solcher zusätzlichen biologischen Ergänzungen, 
    da dies einfach nicht möglich ist durch die Positionierung im Eis. Jedoch findet IceCube auch Anwendung 
    bei der Untersuchung magnetischer Monopole und dient eher bei physikalischen Untersuchungen. \cite{AntOV13}

    \section{Double-Chooz-Experiment}

    Beim Double-Chooz-Experiment handelt es sich um ein Neutrionoteleskop, welches sich durch seine 
    \grqq Eigenschaften\grqq{} wesentlich mehr von IceCube unterscheidet als Antares. Dies beginnt schon bei der 
    Ausrichtung der Forschung. Während bei IceCube nach den Quellen von Neutrinos geforscht wird, wird 
    beim Double-Chooz-Experiment die Neutrinooszillation untersucht. Darunter versteht man die Eigenschaft 
    des Neutrinos sich in eine andere Art des Neutrinos umzuwandeln, wie beispielsweise von einem Antineutrino 
    in ein Myonneutrino. Dieses Experiment wird im Kernkraftwerk Chooz betrieben. Das Ziel besteht darin 
    herauszufinden wie hoch die Wahrscheinlichkeit ist, dass Neutrinooszillation stattfindet, dafür werden 
    zwei Detektoren, in unterschiedlich weitem Abstand vom Reaktor, positioniert. Als Nebenprodukt des 
    inversen radioaktiven $\beta$-Zerfalls entstehen Antineutrinos, die sich willkürlich in alle Richtungen ausbreiten:
    \begin{center}
        $\overline{\nu}_e + p \rightarrow n + e^+ $
    \end{center}
    Man sieht, dass ein Antielektronneutrino und ein Proton in ein Neutron und in ein Positron umwandelt. 
    Die beiden Detektoren registrieren lediglich Antineutrinos. Wenn also der weit entfernte Detektor 
    weniger Messergebnisse hat, kann man daraus schließen, dass sich die Antineutrinos eher umgewandelt 
    haben als im nahen Detektor. Dazu erzeugt das Neutron Szintillationslicht. Dies geschieht indem das 
    Neutron auf ein Elektron des Elements Gadolinium trifft, welches sich in dem Detektor befindet. 
    Das Elektron wird in einen energetisch höheren Zustand gehoben und fällt wieder in seine ursprüngliche 
    Bahn, wodurch Energie in Form von Licht frei wird. Wenn man die Messdaten der beiden Detektoren miteinander 
    vergleicht, lässt sich die Umwandlungswahrscheinlichkeit annäherungsweise bestimmen. Anhand dieser 
    Forschungsart lässt sich erkennen, dass die Ziele der einzelnen Forschungsstätte sich stark unterscheiden, 
    aber trotzdem alle Ergebnisse dazu beitragen die Neutrinos besser zu verstehen. \cite{DCCos12}

    \section{Gallex (Gallium-Experiment)}

    Gallex beschäftigte sich überwiegend mit solaren Neutrinos (also von der Sonne emittierte Neutrinos). Anders als IceCube ist Gallex nicht mehr in Betrieb, 
    da die Ziele der Forschung erreicht wurden. Mit diesem Detektor sollte der Beweis für die Existenz solarer Neutrinos erbracht werden. 
    Mit dem Beweis sollten Theorien zur Energieerzeugung der Sonne bewiesen oder widerlegt werden. Wie der Name schon andeutet handelt 
    es sich um einen Nachweis mit dem Element Gallium. Der Detektor war gefüllt mit einer großen Menge Galliumtrichlorid-Lösung, 
    die zu etwa 30\% aus Gallium besteht. Beim Auftreffen eines Neutrinos gibt es einen sogenannten inversen $\beta$-Zerfall, also eine
    Kernreaktion. Dort entsteht kein Neutrino sondern der Zerfall wird durch ein Neutrino ausgelöst:
    \begin{center}
        $\nu_e+^{71}Ga\rightarrow e^- + ^{71}Ge^+$
    \end{center}
    Aus dem Neutrino und dem Gallium entstehen ein Elektron und Germanium. Gallium wird verwendet, 
    da es eine niedrige Schwellenenergie besitzt, was zur Folge hat, dass auch Neutrinos mit niedrigen 
    Energien einen inversen $\beta$-Zerfall auslösen können. Dies ist sinnvoll, da die solaren Neutrinos eine 
    eher geringe Energie haben. Einige solare Neutrinos erreichen nicht die Erde oder durchdringen deshalb 
    nicht die Erde aufgrund ihrer geringen Energie. Bei Gallex beträgt diese Schwellenenergie etwa 233 keV 
    und bei IceCube liegt diese bei ungefähr 200 GeV. Das ist in etwa ein Unterschied vom Faktor $10^6$. Dies 
    zeigt, dass die solaren Neutrinos eine deutlich geringere Energie aufweisen. Das entstandene Germanium 
    wurde dann extrahiert und in das Gas Monogerman umgewandelt. Dieses hat eine recht kurze Halbwertszeit 
    (~11,4 Tage), mit der man nach jedem Zerfall ein Neutrino \grqq eingefangen\grqq{} hatte. Durch diesen Detektor konnte 
    man den ersten Nachweis dafür erbringen, dass Neutrinos oszillieren, denn mathematische Modelle sagten 
    mehr Registrierungen von Neutrinos hervor. Allerdings registriert dieser Detektor lediglich 
    Elektronenneutrinos, was bedeutet, dass die Neutrinos oszilliert haben. Durch die Oszillation wurden die 
    Elektronenneutrinos in andere Neutrinos umgewandelt, die der Detektor nicht registriert. Eine Bedingung 
    für die Oszillation ist, dass die Neutrinos Masse haben, was dadurch bewiesen wurde. Vorher nahm man an, 
    dass Neutrinos masselos sind, was durch diese Experimente widerlegt wurde. IceCube weißt ebenfalls große 
    Erfolge auf, wie die Entdeckung von Neutrinoquellen. Allerdings legen solche Entdeckungen den Grundbaustein 
    für nachfolgende Projekte wie IceCube. \cite{PhyGA13}

    \section{Ergebnisse des Vergleichs}

    Abschließend, nach einem ausführlichen Vergleich, stellt sich die Frage, ob es sinnvoll ist alle 
    Detektoren weiter zu betrieben oder ob nur bestimmte weiter betrieben und mit weiteren Fördermitteln 
    subventioniert werden sollen. Zuerst sollte man beachten, dass alle genannten Observatorien wichtige 
    Daten gesammelt haben, die dazu beitragen die Neutrinos besser zu verstehen. Zum anderen ist es sinnvoll 
    möglichst viele Daten zu sammeln, da die Wechselwirkung zwischen kosmischen Neutrino und Materie selten 
    stattfindet. Deshalb ist es nützlich möglichst viele unterschiedliche Nachweismedien zu verwenden, um zu 
    testen, welches das mit der häufigsten Interaktionsquote ist, um die zukünftig Forschung zu verbessern. 
    Nicht  zu  vergessen  ist,  dass  sich  die  Ziele,  die  man  erreichen  und  die Grundlagen, die man 
    versucht zu erklären unterschiedlich sind. Die Forschung richtet sich zwar nach den Neutrinos, aber 
    spezifisch beschäftigen ich die Kollaborationen mit unterschiedlichen Schwerpunkten und Kollaborationen 
    mit den selben Schwerpunkten haben die Möglichkeit einer engen Kooperation, um ihre Ergebnisse zu 
    vergleichen und effizienter zu arbeiten und zu forschen. Außerdem muss man bedenken, dass viele Fragen 
    in einem geringen Zeitraum beantwortet werden können und nicht alle in einem großen zeitlichen Abstand 
    beantwortet werden müssen. Nicht zu vergessen ist außerdem, dass bestimmte Daten anderen 
    Forschungseinrichtungen dabei helfen können ihre Forschung zu optimieren und anzupassen. Deshalb ist es 
    logisch, dass mehrere Forschungsprojekte zur selben Zeit laufen. Selbstverständlich müssen diese 
    Forschungseinrichtungen irgendwann außer Betrieb gesetzt werden, da diese sonst nur weitere Gelder 
    beanspruchen würden, die an anderen Stellen sinnvoller sein könnten. Dies sieht man beispielsweise 
    gut an Gallex. Gallex hat die aufgeworfenen Fragen beantwortet. Eine Fortführung wäre überflüssig, 
    da sie keine weiteren Antworten mehr liefern könnte. Anders verhält es sich beispielsweise bei IceCube. 
    Während man mit Gallex einige Theorien klären sollte, versucht man mit IceCube auch Quellen zu 
    identifizieren, die uns weiter helfen könnten und es gibt sehr viele Quellen, die man noch lokalisieren 
    könnte. Außerdem geht IceCube nun weit über seine eigentlichen Ziele hinaus. Mit IceCube gibt es noch 
    viele Forschungsmöglichkeiten. Als Beispiel kann man anführen, dass jetzt nach den GZK-Neutrinos 
    geforscht werden soll. Das Prinzip von IceCube ermöglicht theoretisch die Registrierung dieser Neutrinos. 
    Die Wahrscheinlichkeit, dass diese Neutrinos mit dem Eis wechselwirken ist sehr gering. Man muss bedenken, 
    wie viele Millionen Neutrinos uns pro Sekunde durchqueren. Der prozentuale Anteil davon wie viele davon mit 
    Materie wechselwirken ist gering. Nun muss man bedenken, dass es vermutlich mehr normale kosmische 
    Neutrinos gibt als GZK-Neutrinos, was es ebenfalls wieder unwahrscheinlicher macht, dass sie mit 
    Materie wechselwirken. Deshalb ist es umso wichtiger mehr Daten zu sammeln, was vor allem mit IceCube 
    möglich ist, da IceCube das größte Volumen der Neutrinoteilchendetektoren darstellt. Wenn sich nun die 
    Frage stellt, ob man nun Antares oder IceCube neue Fördermittel zur Verfügung stellt, wäre es natürlich 
    wünschenswert, das beide mit neuen Gelder ausgestattet werden, denn beide würden die Neutrinoforschung 
    voranbringen. Wenn dies jedoch nicht möglich wäre und eine Entscheidung gefällt werden muss, dann würde 
    ich es am sinnvollsten erachten, dass IceCube eher mit neunen Fördergeldern unterstützt wird. Dies halte 
    ich zum einen für sinnvoll, da IceCube schon gebaut wurde und Antares müsste noch gebaut werden und ist 
    dann in etwa vergleichbar mit IceCube, aber ohne die Erweiterung, denn im Prinzip sind es die selben 
    Detektoren, welche sich lediglich beide in$H_2O$ befinden, nur in einem anderen Aggregatzustand. Außerdem 
    hat die IceCube-Kollaboration deutlich mehr Erfahrung in der Neutrinoforschung und weiß wie man am besten
    vorgeht und sind schon eher in dem Denken drin, wie man das Projekt noch optimieren könnte oder wie man die 
    Daten am besten analysiert. Nicht zu vergessen ist, dass IceCube schon konkrete Pläne hat, wie man den 
    Detektor am besten ausbaut um auch in anderen Gebieten der Neutrinos zu forschen, hier lässt sich auf die 
    PINGU-Erweiterung verweisen, die zwar nicht gebaut wurde, aber immer noch geplant wird, wenn Fördergelder 
    bereitgestellt werden. \cite{Neutrino14}

   
    \vfill  
    \pagebreak  
    \end{doublespace}    
    
    %
    % Literaturverzeichnis 
    \addcontentsline{toc}{chapter}{Literaturverzeichnis}
    %
    \bibliography{bibliography}
    \bibliographystyle{alpha}

    \appendix
    \chapter{Anhänge}

    

    \section{Beta-Zerfall Spektrum}
    \label{fig:b-zerfall}
    \begin{center}
        \includegraphics[width=1\textwidth]{appendix/Tritiumzerfall-elektronenergiespektrum}
    \end{center}
    Credits: G. M. Lewis, Neutrinos (London: Wykeham, 1970) Seite 30.

    \section{Brief: Paulis Vorschlag von Neutrinos}
    \label{lab:pauli-brief}    
    Offener Brief an die Gruppe der Radioaktiven bei der Gauverein-Tagung zu Tübingen. \\
Physikalisches Institut \\
der Eidg. Technischen Hochschule\\
Zürch \\
Zürich, 4. Dez. 1930 \\
Gloriastrasse \\
Liebe radioaktive Damen und Herren, 
Wie der Ueberbringer dieser Zeilen, den ich huldvollst anzuhöhren bitte, 
Ihnen des näheren auseinandersetzen wird, bin ich angesichts der ``falschen'' 
Statstik der N- und Li-6 Kerne, sowie des kontinuierlchen beta-Spektrums auf 
einen verzweifelten Ausweg verfallen um den ``Wechselsatz'' (1) der Statistik 
und den Energiesatz zu retten. Nämlich die Möglichkeit, es könnten elektrisch 
neutrale Teilchen, die ich Neutronen nennen will, in den Kernen existieren, 
welche den Spin 1/2 haben und das Ausschliessungsprinzip befolgen und sich von 
Lichtquanten ausserdem noch dadurch unterscheiden, dass sie nicht mit Lichtgeschwindigkeit 
laufen. Die Masse der Neutronen müsste von derselben Größenordnung wie die Elektronenmassen
sein und jedenfalls nicht größer als 0,01 Protonenmasse. Das kontinuierliche beta-Spektrum 
wäre dann verständlich unter der Annahme, das beta-Zerfall mit dem Elektron jeweils noch 
ein Neutron emittiert würde derart, dass die Summe der Energien von Neutronen und Elekronen 
konstant ist. \\ \\
Nun handelt es sich weiter darum, welche Kräfte auf die Neutronen wirken. 
Das wahrscheinlichste Modell für das Neutron scheint mir aus wellenmechanischen Gründen 
(näheres weiss der Ueberbringer diese Zeilen) dieses zu sein, dass das ruhende Neutron ein 
magnetischer Dipol von einem gewissen Moment $\mu$ ist. Die Experimente verlangen wohl, 
dass die ionisierende Wirkung eines solchen Neutrons nicht grösser sein kann, 
als die eines gamma-Strahls und darf dann $\mu$ wohl nicht grösser sein als e*(10-13 cm). \\ \\
Ich traue mich verläufig aber nicht, etwas über diese Idee zu publizieren und wende mich 
erst vertrauensvoll an Euch, liebe Radioaktive, mit der Frage, wie es um den 
experimentellen Nachweis eines solchen Neutrons stände, wenn dieses ein ebensolches oder 
etwa 10mal größeres Durchdringungsvermögen besitzen würde, wie ein gamma-Strahl. \\ \\
Ich gebe zu, dass mein Ausweg vielleicht von vornherein wenig wahrscheinlich 
erscheinen wird, weil man die Neutronen, wenn sie existieren, wohl längst gesehen hätte. 
Aber nur wer wagt, gewinnt und der Ernst der Situation beim kontinuierlichen beta-Spektrum 
wir durch den Ausspruch meines verehrten Vorgängers im Amte, Herrn Debye, beleuchtet, der 
mir kürzlich in Brüssel gesagt hat: ``O, daran soll man am besten gar nicht denken, sowie 
an die neuen Steuern.'' Darum soll man jeden Weg zur Rettung ernstlich diskutieren. Also 
liebe Radioaktive, prüfet, und richtet. - Leider kann ich nicht persönlich in Tübingen 
erscheinen, da ich infolge eines in der Nacht vom 6. zum 7. Dez. in Zürich stattfindenden 
Balles hier unabkömmlich bin. - Mit vielen Grüssen an Euch, sowie auch an Herrn Back, 
Euer untertänigster Dienser \\ \\ 
gez. W. Pauli

    \section{Cowan-Reines Experiment}
    \label{fig:c-experiment}
    \begin{center}
        \includegraphics[width=1\textwidth]{appendix/cowanreines}
    \end{center}   
    Quelle: timaios.org

    \section{DUNE}
    \label{fig:dune}
    \begin{center}
        \includegraphics[width=1\textwidth]{appendix/dune}
    \end{center}
    Quelle: Dune-Kollaboration
    
    \end{document}
    %%%%%%%%%%%%%%%%%%%%%%%%%%%%%%%%%%%%%%%%%%%%%%%%%%%%%%%%%